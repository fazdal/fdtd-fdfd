
% Default to the notebook output style

    


% Inherit from the specified cell style.




    
\documentclass{article}

    
    
    \usepackage{graphicx} % Used to insert images
    \usepackage{adjustbox} % Used to constrain images to a maximum size 
    \usepackage{color} % Allow colors to be defined
    \usepackage{enumerate} % Needed for markdown enumerations to work
    \usepackage{geometry} % Used to adjust the document margins
    \usepackage{amsmath} % Equations
    \usepackage{amssymb} % Equations
    \usepackage{eurosym} % defines \euro
    \usepackage[mathletters]{ucs} % Extended unicode (utf-8) support
    \usepackage[utf8x]{inputenc} % Allow utf-8 characters in the tex document
    \usepackage{fancyvrb} % verbatim replacement that allows latex
    \usepackage{grffile} % extends the file name processing of package graphics 
                         % to support a larger range 
    % The hyperref package gives us a pdf with properly built
    % internal navigation ('pdf bookmarks' for the table of contents,
    % internal cross-reference links, web links for URLs, etc.)
    \usepackage{hyperref}
    \usepackage{longtable} % longtable support required by pandoc >1.10
    \usepackage{booktabs}  % table support for pandoc > 1.12.2
    \usepackage{ulem} % ulem is needed to support strikethroughs (\sout)
    

    
    
    \definecolor{orange}{cmyk}{0,0.4,0.8,0.2}
    \definecolor{darkorange}{rgb}{.71,0.21,0.01}
    \definecolor{darkgreen}{rgb}{.12,.54,.11}
    \definecolor{myteal}{rgb}{.26, .44, .56}
    \definecolor{gray}{gray}{0.45}
    \definecolor{lightgray}{gray}{.95}
    \definecolor{mediumgray}{gray}{.8}
    \definecolor{inputbackground}{rgb}{.95, .95, .85}
    \definecolor{outputbackground}{rgb}{.95, .95, .95}
    \definecolor{traceback}{rgb}{1, .95, .95}
    % ansi colors
    \definecolor{red}{rgb}{.6,0,0}
    \definecolor{green}{rgb}{0,.65,0}
    \definecolor{brown}{rgb}{0.6,0.6,0}
    \definecolor{blue}{rgb}{0,.145,.698}
    \definecolor{purple}{rgb}{.698,.145,.698}
    \definecolor{cyan}{rgb}{0,.698,.698}
    \definecolor{lightgray}{gray}{0.5}
    
    % bright ansi colors
    \definecolor{darkgray}{gray}{0.25}
    \definecolor{lightred}{rgb}{1.0,0.39,0.28}
    \definecolor{lightgreen}{rgb}{0.48,0.99,0.0}
    \definecolor{lightblue}{rgb}{0.53,0.81,0.92}
    \definecolor{lightpurple}{rgb}{0.87,0.63,0.87}
    \definecolor{lightcyan}{rgb}{0.5,1.0,0.83}
    
    % commands and environments needed by pandoc snippets
    % extracted from the output of `pandoc -s`
    \providecommand{\tightlist}{%
      \setlength{\itemsep}{0pt}\setlength{\parskip}{0pt}}
    \DefineVerbatimEnvironment{Highlighting}{Verbatim}{commandchars=\\\{\}}
    % Add ',fontsize=\small' for more characters per line
    \newenvironment{Shaded}{}{}
    \newcommand{\KeywordTok}[1]{\textcolor[rgb]{0.00,0.44,0.13}{\textbf{{#1}}}}
    \newcommand{\DataTypeTok}[1]{\textcolor[rgb]{0.56,0.13,0.00}{{#1}}}
    \newcommand{\DecValTok}[1]{\textcolor[rgb]{0.25,0.63,0.44}{{#1}}}
    \newcommand{\BaseNTok}[1]{\textcolor[rgb]{0.25,0.63,0.44}{{#1}}}
    \newcommand{\FloatTok}[1]{\textcolor[rgb]{0.25,0.63,0.44}{{#1}}}
    \newcommand{\CharTok}[1]{\textcolor[rgb]{0.25,0.44,0.63}{{#1}}}
    \newcommand{\StringTok}[1]{\textcolor[rgb]{0.25,0.44,0.63}{{#1}}}
    \newcommand{\CommentTok}[1]{\textcolor[rgb]{0.38,0.63,0.69}{\textit{{#1}}}}
    \newcommand{\OtherTok}[1]{\textcolor[rgb]{0.00,0.44,0.13}{{#1}}}
    \newcommand{\AlertTok}[1]{\textcolor[rgb]{1.00,0.00,0.00}{\textbf{{#1}}}}
    \newcommand{\FunctionTok}[1]{\textcolor[rgb]{0.02,0.16,0.49}{{#1}}}
    \newcommand{\RegionMarkerTok}[1]{{#1}}
    \newcommand{\ErrorTok}[1]{\textcolor[rgb]{1.00,0.00,0.00}{\textbf{{#1}}}}
    \newcommand{\NormalTok}[1]{{#1}}
    
    % Additional commands for more recent versions of Pandoc
    \newcommand{\ConstantTok}[1]{\textcolor[rgb]{0.53,0.00,0.00}{{#1}}}
    \newcommand{\SpecialCharTok}[1]{\textcolor[rgb]{0.25,0.44,0.63}{{#1}}}
    \newcommand{\VerbatimStringTok}[1]{\textcolor[rgb]{0.25,0.44,0.63}{{#1}}}
    \newcommand{\SpecialStringTok}[1]{\textcolor[rgb]{0.73,0.40,0.53}{{#1}}}
    \newcommand{\ImportTok}[1]{{#1}}
    \newcommand{\DocumentationTok}[1]{\textcolor[rgb]{0.73,0.13,0.13}{\textit{{#1}}}}
    \newcommand{\AnnotationTok}[1]{\textcolor[rgb]{0.38,0.63,0.69}{\textbf{\textit{{#1}}}}}
    \newcommand{\CommentVarTok}[1]{\textcolor[rgb]{0.38,0.63,0.69}{\textbf{\textit{{#1}}}}}
    \newcommand{\VariableTok}[1]{\textcolor[rgb]{0.10,0.09,0.49}{{#1}}}
    \newcommand{\ControlFlowTok}[1]{\textcolor[rgb]{0.00,0.44,0.13}{\textbf{{#1}}}}
    \newcommand{\OperatorTok}[1]{\textcolor[rgb]{0.40,0.40,0.40}{{#1}}}
    \newcommand{\BuiltInTok}[1]{{#1}}
    \newcommand{\ExtensionTok}[1]{{#1}}
    \newcommand{\PreprocessorTok}[1]{\textcolor[rgb]{0.74,0.48,0.00}{{#1}}}
    \newcommand{\AttributeTok}[1]{\textcolor[rgb]{0.49,0.56,0.16}{{#1}}}
    \newcommand{\InformationTok}[1]{\textcolor[rgb]{0.38,0.63,0.69}{\textbf{\textit{{#1}}}}}
    \newcommand{\WarningTok}[1]{\textcolor[rgb]{0.38,0.63,0.69}{\textbf{\textit{{#1}}}}}
    
    
    % Define a nice break command that doesn't care if a line doesn't already
    % exist.
    \def\br{\hspace*{\fill} \\* }
    % Math Jax compatability definitions
    \def\gt{>}
    \def\lt{<}
    % Document parameters
    \title{Task 2: Convolutional perfectly matched
layer}
    \author{Ang, Angeleene S.}
    
    

    % Pygments definitions
    
\makeatletter
\def\PY@reset{\let\PY@it=\relax \let\PY@bf=\relax%
    \let\PY@ul=\relax \let\PY@tc=\relax%
    \let\PY@bc=\relax \let\PY@ff=\relax}
\def\PY@tok#1{\csname PY@tok@#1\endcsname}
\def\PY@toks#1+{\ifx\relax#1\empty\else%
    \PY@tok{#1}\expandafter\PY@toks\fi}
\def\PY@do#1{\PY@bc{\PY@tc{\PY@ul{%
    \PY@it{\PY@bf{\PY@ff{#1}}}}}}}
\def\PY#1#2{\PY@reset\PY@toks#1+\relax+\PY@do{#2}}

\expandafter\def\csname PY@tok@no\endcsname{\def\PY@tc##1{\textcolor[rgb]{0.53,0.00,0.00}{##1}}}
\expandafter\def\csname PY@tok@gr\endcsname{\def\PY@tc##1{\textcolor[rgb]{1.00,0.00,0.00}{##1}}}
\expandafter\def\csname PY@tok@w\endcsname{\def\PY@tc##1{\textcolor[rgb]{0.73,0.73,0.73}{##1}}}
\expandafter\def\csname PY@tok@mo\endcsname{\def\PY@tc##1{\textcolor[rgb]{0.40,0.40,0.40}{##1}}}
\expandafter\def\csname PY@tok@s2\endcsname{\def\PY@tc##1{\textcolor[rgb]{0.73,0.13,0.13}{##1}}}
\expandafter\def\csname PY@tok@kp\endcsname{\def\PY@tc##1{\textcolor[rgb]{0.00,0.50,0.00}{##1}}}
\expandafter\def\csname PY@tok@cm\endcsname{\let\PY@it=\textit\def\PY@tc##1{\textcolor[rgb]{0.25,0.50,0.50}{##1}}}
\expandafter\def\csname PY@tok@ge\endcsname{\let\PY@it=\textit}
\expandafter\def\csname PY@tok@k\endcsname{\let\PY@bf=\textbf\def\PY@tc##1{\textcolor[rgb]{0.00,0.50,0.00}{##1}}}
\expandafter\def\csname PY@tok@nc\endcsname{\let\PY@bf=\textbf\def\PY@tc##1{\textcolor[rgb]{0.00,0.00,1.00}{##1}}}
\expandafter\def\csname PY@tok@ss\endcsname{\def\PY@tc##1{\textcolor[rgb]{0.10,0.09,0.49}{##1}}}
\expandafter\def\csname PY@tok@bp\endcsname{\def\PY@tc##1{\textcolor[rgb]{0.00,0.50,0.00}{##1}}}
\expandafter\def\csname PY@tok@mi\endcsname{\def\PY@tc##1{\textcolor[rgb]{0.40,0.40,0.40}{##1}}}
\expandafter\def\csname PY@tok@sh\endcsname{\def\PY@tc##1{\textcolor[rgb]{0.73,0.13,0.13}{##1}}}
\expandafter\def\csname PY@tok@gs\endcsname{\let\PY@bf=\textbf}
\expandafter\def\csname PY@tok@gi\endcsname{\def\PY@tc##1{\textcolor[rgb]{0.00,0.63,0.00}{##1}}}
\expandafter\def\csname PY@tok@mf\endcsname{\def\PY@tc##1{\textcolor[rgb]{0.40,0.40,0.40}{##1}}}
\expandafter\def\csname PY@tok@nl\endcsname{\def\PY@tc##1{\textcolor[rgb]{0.63,0.63,0.00}{##1}}}
\expandafter\def\csname PY@tok@sb\endcsname{\def\PY@tc##1{\textcolor[rgb]{0.73,0.13,0.13}{##1}}}
\expandafter\def\csname PY@tok@nf\endcsname{\def\PY@tc##1{\textcolor[rgb]{0.00,0.00,1.00}{##1}}}
\expandafter\def\csname PY@tok@nt\endcsname{\let\PY@bf=\textbf\def\PY@tc##1{\textcolor[rgb]{0.00,0.50,0.00}{##1}}}
\expandafter\def\csname PY@tok@ne\endcsname{\let\PY@bf=\textbf\def\PY@tc##1{\textcolor[rgb]{0.82,0.25,0.23}{##1}}}
\expandafter\def\csname PY@tok@sx\endcsname{\def\PY@tc##1{\textcolor[rgb]{0.00,0.50,0.00}{##1}}}
\expandafter\def\csname PY@tok@il\endcsname{\def\PY@tc##1{\textcolor[rgb]{0.40,0.40,0.40}{##1}}}
\expandafter\def\csname PY@tok@vi\endcsname{\def\PY@tc##1{\textcolor[rgb]{0.10,0.09,0.49}{##1}}}
\expandafter\def\csname PY@tok@na\endcsname{\def\PY@tc##1{\textcolor[rgb]{0.49,0.56,0.16}{##1}}}
\expandafter\def\csname PY@tok@nv\endcsname{\def\PY@tc##1{\textcolor[rgb]{0.10,0.09,0.49}{##1}}}
\expandafter\def\csname PY@tok@gt\endcsname{\def\PY@tc##1{\textcolor[rgb]{0.00,0.27,0.87}{##1}}}
\expandafter\def\csname PY@tok@gu\endcsname{\let\PY@bf=\textbf\def\PY@tc##1{\textcolor[rgb]{0.50,0.00,0.50}{##1}}}
\expandafter\def\csname PY@tok@kr\endcsname{\let\PY@bf=\textbf\def\PY@tc##1{\textcolor[rgb]{0.00,0.50,0.00}{##1}}}
\expandafter\def\csname PY@tok@c\endcsname{\let\PY@it=\textit\def\PY@tc##1{\textcolor[rgb]{0.25,0.50,0.50}{##1}}}
\expandafter\def\csname PY@tok@nn\endcsname{\let\PY@bf=\textbf\def\PY@tc##1{\textcolor[rgb]{0.00,0.00,1.00}{##1}}}
\expandafter\def\csname PY@tok@sr\endcsname{\def\PY@tc##1{\textcolor[rgb]{0.73,0.40,0.53}{##1}}}
\expandafter\def\csname PY@tok@sc\endcsname{\def\PY@tc##1{\textcolor[rgb]{0.73,0.13,0.13}{##1}}}
\expandafter\def\csname PY@tok@c1\endcsname{\let\PY@it=\textit\def\PY@tc##1{\textcolor[rgb]{0.25,0.50,0.50}{##1}}}
\expandafter\def\csname PY@tok@kd\endcsname{\let\PY@bf=\textbf\def\PY@tc##1{\textcolor[rgb]{0.00,0.50,0.00}{##1}}}
\expandafter\def\csname PY@tok@s1\endcsname{\def\PY@tc##1{\textcolor[rgb]{0.73,0.13,0.13}{##1}}}
\expandafter\def\csname PY@tok@cpf\endcsname{\let\PY@it=\textit\def\PY@tc##1{\textcolor[rgb]{0.25,0.50,0.50}{##1}}}
\expandafter\def\csname PY@tok@s\endcsname{\def\PY@tc##1{\textcolor[rgb]{0.73,0.13,0.13}{##1}}}
\expandafter\def\csname PY@tok@vc\endcsname{\def\PY@tc##1{\textcolor[rgb]{0.10,0.09,0.49}{##1}}}
\expandafter\def\csname PY@tok@mh\endcsname{\def\PY@tc##1{\textcolor[rgb]{0.40,0.40,0.40}{##1}}}
\expandafter\def\csname PY@tok@cs\endcsname{\let\PY@it=\textit\def\PY@tc##1{\textcolor[rgb]{0.25,0.50,0.50}{##1}}}
\expandafter\def\csname PY@tok@si\endcsname{\let\PY@bf=\textbf\def\PY@tc##1{\textcolor[rgb]{0.73,0.40,0.53}{##1}}}
\expandafter\def\csname PY@tok@ch\endcsname{\let\PY@it=\textit\def\PY@tc##1{\textcolor[rgb]{0.25,0.50,0.50}{##1}}}
\expandafter\def\csname PY@tok@err\endcsname{\def\PY@bc##1{\setlength{\fboxsep}{0pt}\fcolorbox[rgb]{1.00,0.00,0.00}{1,1,1}{\strut ##1}}}
\expandafter\def\csname PY@tok@gh\endcsname{\let\PY@bf=\textbf\def\PY@tc##1{\textcolor[rgb]{0.00,0.00,0.50}{##1}}}
\expandafter\def\csname PY@tok@go\endcsname{\def\PY@tc##1{\textcolor[rgb]{0.53,0.53,0.53}{##1}}}
\expandafter\def\csname PY@tok@ow\endcsname{\let\PY@bf=\textbf\def\PY@tc##1{\textcolor[rgb]{0.67,0.13,1.00}{##1}}}
\expandafter\def\csname PY@tok@vg\endcsname{\def\PY@tc##1{\textcolor[rgb]{0.10,0.09,0.49}{##1}}}
\expandafter\def\csname PY@tok@nb\endcsname{\def\PY@tc##1{\textcolor[rgb]{0.00,0.50,0.00}{##1}}}
\expandafter\def\csname PY@tok@nd\endcsname{\def\PY@tc##1{\textcolor[rgb]{0.67,0.13,1.00}{##1}}}
\expandafter\def\csname PY@tok@gd\endcsname{\def\PY@tc##1{\textcolor[rgb]{0.63,0.00,0.00}{##1}}}
\expandafter\def\csname PY@tok@kt\endcsname{\def\PY@tc##1{\textcolor[rgb]{0.69,0.00,0.25}{##1}}}
\expandafter\def\csname PY@tok@m\endcsname{\def\PY@tc##1{\textcolor[rgb]{0.40,0.40,0.40}{##1}}}
\expandafter\def\csname PY@tok@cp\endcsname{\def\PY@tc##1{\textcolor[rgb]{0.74,0.48,0.00}{##1}}}
\expandafter\def\csname PY@tok@se\endcsname{\let\PY@bf=\textbf\def\PY@tc##1{\textcolor[rgb]{0.73,0.40,0.13}{##1}}}
\expandafter\def\csname PY@tok@kc\endcsname{\let\PY@bf=\textbf\def\PY@tc##1{\textcolor[rgb]{0.00,0.50,0.00}{##1}}}
\expandafter\def\csname PY@tok@o\endcsname{\def\PY@tc##1{\textcolor[rgb]{0.40,0.40,0.40}{##1}}}
\expandafter\def\csname PY@tok@ni\endcsname{\let\PY@bf=\textbf\def\PY@tc##1{\textcolor[rgb]{0.60,0.60,0.60}{##1}}}
\expandafter\def\csname PY@tok@sd\endcsname{\let\PY@it=\textit\def\PY@tc##1{\textcolor[rgb]{0.73,0.13,0.13}{##1}}}
\expandafter\def\csname PY@tok@gp\endcsname{\let\PY@bf=\textbf\def\PY@tc##1{\textcolor[rgb]{0.00,0.00,0.50}{##1}}}
\expandafter\def\csname PY@tok@mb\endcsname{\def\PY@tc##1{\textcolor[rgb]{0.40,0.40,0.40}{##1}}}
\expandafter\def\csname PY@tok@kn\endcsname{\let\PY@bf=\textbf\def\PY@tc##1{\textcolor[rgb]{0.00,0.50,0.00}{##1}}}

\def\PYZbs{\char`\\}
\def\PYZus{\char`\_}
\def\PYZob{\char`\{}
\def\PYZcb{\char`\}}
\def\PYZca{\char`\^}
\def\PYZam{\char`\&}
\def\PYZlt{\char`\<}
\def\PYZgt{\char`\>}
\def\PYZsh{\char`\#}
\def\PYZpc{\char`\%}
\def\PYZdl{\char`\$}
\def\PYZhy{\char`\-}
\def\PYZsq{\char`\'}
\def\PYZdq{\char`\"}
\def\PYZti{\char`\~}
% for compatibility with earlier versions
\def\PYZat{@}
\def\PYZlb{[}
\def\PYZrb{]}
\makeatother


    % Exact colors from NB
    \definecolor{incolor}{rgb}{0.0, 0.0, 0.5}
    \definecolor{outcolor}{rgb}{0.545, 0.0, 0.0}



    
    % Prevent overflowing lines due to hard-to-break entities
    \sloppy 
    % Setup hyperref package
    \hypersetup{
      breaklinks=true,  % so long urls are correctly broken across lines
      colorlinks=true,
      urlcolor=blue,
      linkcolor=darkorange,
      citecolor=darkgreen,
      }
    % Slightly bigger margins than the latex defaults
    
    \geometry{verbose,tmargin=1in,bmargin=1in,lmargin=1in,rmargin=1in}
    
    

    \begin{document}
    
    
    \maketitle
    
    

    
    \subsection*{Derivation}\label{derivation}

An alternative method in creating a simulation space with seemingly
endless boundaries is the perfectly matched layer (PML). These layers
are created such that any incident wave propagating inside the PML is
absorbed, and another requirement is that the PML must have zero
reflection at the interface.

The process in which this happens is as follows: An incident wave will
initially enter the PML domain, and will be attenuated by a rate
\(\alpha_{PML}\) over the thickness of the layer \(\Delta PML\), and
then be reflected at the edge of the simulation space, which has a
perfect electric conductor.

After reflection, the wave will still be inside the PML domain, so it
will be attenuated again at a rate \(\alpha_{PML}\) in the layer. As a
result, the wave will be attenuated by a factor
\(e^{-2 \alpha_{PML} \Delta PML}\). This reflection factor, in the more
general 2D case, is of the form

\begin{equation}
R(\theta) = e^{-2 \eta \cos \theta \int_{0}^{\Delta PML} \sigma(x) \mathrm{d} x }
\end{equation}

however, in 1D, the incident wave is always at normal incidence to the
boundary. We can take the reflection factor to be a constant

\begin{equation}
R(0) = 10^{-6}
\end{equation}

    In order to reduce the reflection error from the discretization of the
domain, Bérenger introduced grading the PML conductivity \(\sigma\)
smoothly, from zero to some maximum value at the boundary. One variation
of this grading is the polynomial grading, in which the conductivity
inside the PML is given by

\begin{equation}
\sigma(x) = \sigma_{max} \left( \dfrac{x}{\Delta PML} \right)^m 
\end{equation}

Here, \(m\) is the order of polynomial grading; \(3 \leq m \leq 4\) has
been found to be optimal for FDTD, and the maximum conductivity
\(\sigma_{max}\) is given by

\begin{equation}
\sigma_{max} = -\dfrac{(m+1) \ln[R(0)]}{2\eta \Delta PML}
\end{equation}

    In this task, we use the complex-frequency-shifted (CFS) PML method
derived by Roden and Gedney. We start with Ampere's Law in the frequency
domain

\begin{equation}
i \omega \epsilon E_z = \dfrac{1}{s_E} \dfrac{\partial H_y}{\partial x}
\end{equation}

in which the material where field propagates in outside the PML has zero
losses, and where \(s_E\) is some constant given by

\begin{equation}
s_E = 1 + \dfrac{\sigma}{i \omega \epsilon_0}
\end{equation}

The subscript \(E\) indicates that this \(s\) is related to the electric
field. In the code, \(s_E\) and \(s_H\) have a small shift due to how
the grid is defined. Transforming the equation for \(E_z\) into the time
domain,

\begin{equation}
\epsilon \dfrac{\partial E_z}{\partial t} = \bar{s_E} * \dfrac{\partial H_y}{\partial x}
\end{equation}

where \(\bar{s_E}\) is defined as the Laplace transform of the original
\(s_E\), and \(*\) indicates the convolution operator. The function
\(\bar{s_E}\) is given by

\begin{equation}
\bar{s_E} = \delta(t) + \zeta_E(t)
\end{equation}

hence

\begin{equation}
\epsilon \dfrac{\partial E_z}{\partial t} = \dfrac{\partial H_y}{\partial x} + \zeta_E(t) * \dfrac{\partial H_y}{\partial x}
\end{equation}

    It can be shown that the discrete response of \(\zeta_E (t)\) is given
by

\begin{equation}
Z_E (m) = a_E \; exp\left( \dfrac{\sigma m \Delta t}{\epsilon_0} \right)
\end{equation}

where the constant \(a_E\) is

\begin{equation}
a_E = \mathrm{exp}\left ( - \dfrac{\sigma \Delta t}{\epsilon_0} \right ) -1 
\end{equation}

    Using the discretization of the derivatives, the update equation based
on Ampere's Law now becomes

\begin{equation}
\epsilon \dfrac{1}{\Delta t} \left ( E_z \Big |^{n+1}_{i} - E_z \Big |^{n}_{i} \right ) = \dfrac{1}{\Delta x} \left ( H_y \Big |^{n+1/2}_{i+1/2} - H_y \Big |^{n+1/2}_{i-1/2} \right ) + \sum^{N-1}_{m=0} \dfrac{1}{\Delta x} Z_E (m) \left ( H_y \Big |^{n-m+1/2}_{i+1/2} - H_y \Big |^{n-m+1/2}_{i-1/2} \right )
\end{equation}

from the form of \(Z_E\), the convolution can be performed in a simpler
manner if we introduced a variable \(\psi_E\) such that we obtain an
update equation

\begin{equation}
\epsilon \dfrac{1}{\Delta t} \left ( E_z \Big |^{n+1}_{i} - E_z \Big |^{n}_{i} \right ) = \dfrac{1}{\Delta x} \left ( H_y \Big |^{n+1/2}_{i+1/2} - H_y \Big |^{n+1/2}_{i-1/2} \right ) + \psi_E \Big |^{n+1/2}_{i+1/2}
\end{equation}

The subscript indicates that this is the \(\psi\) associated with
updating the electric field. This variable is defined such that

\begin{equation}
\psi_E \Big |^{n+1/2}_{i+1/2} = b_E \psi_E \Big |^{n-1/2}_{i+1/2} + \dfrac{a_E}{\Delta x} \left ( H_y \Big |^{n+1/2}_{i+1/2} - H_y \Big |^{n+1/2}_{i-1/2} \right )
\end{equation}

where the constant \(b_E\) defined as

\begin{equation}
b_E = \mathrm{exp}\left ( - \dfrac{\sigma \Delta t}{\epsilon_0} \right )
\end{equation}

Solving thre update equation for \(E_z \Big |^{n+1}_{i+1/2}\),

\begin{equation}
E_z \Big |^{n+1}_{i} = E_z \Big |^{n}_{i} + \dfrac{1}{\epsilon} \dfrac{\Delta t}{\Delta x} \left ( H_y \Big |^{n+1/2}_{i+1/2} - H_y \Big |^{n+1/2}_{i-1/2} \right ) + \Delta t \dfrac{1}{\epsilon} \psi_E \Big |^{n+1/2}_{i+1/2}
\end{equation}

    Using a similar expression from Faraday's Law

\begin{equation}
i \omega \mu H_y = \dfrac{1}{s_H} \dfrac{\partial E_z}{\partial x}
\end{equation}

we obtain the following update equation

\begin{equation}
\mu \dfrac{1}{\Delta t} \left ( H_y \Big |^{n+1/2}_{i+1/2} - H_y \Big |^{n-1/2}_{i+1/2} \right ) = \dfrac{1}{\Delta x} \left ( E_z \Big |^{n}_{i+1} - E_z \Big |^{n}_{i} \right ) + \psi_H \Big |^{n}_{i+1}
\end{equation}

in which \(\psi_H\), \(a_H\), and \(b_H\) are defined similarly

\begin{equation}
\psi_H \Big |^{n}_{i+1} = b_H \psi_H \Big |^{n-1/2}_{i+1/2} + \dfrac{a_H}{\Delta x} \left ( E_z \Big |^{n}_{i+1} - E_z \Big |^{n}_{i} \right )
\end{equation}

\begin{equation}
a_H = \mathrm{exp}\left ( - \dfrac{\sigma \Delta t}{\epsilon_0} \right ) -1 
\end{equation}

\begin{equation}
b_H = \mathrm{exp}\left ( - \dfrac{\sigma \Delta t}{\epsilon_0} \right )
\end{equation}

So the update equation is now

\begin{equation}
H_y \Big |^{n+1/2}_{i+1/2} =  H_y \Big |^{n-1/2}_{i+1/2} + \dfrac{1}{\mu} \dfrac{\Delta t}{\Delta x} \left ( E_z \Big |^{n}_{i+1} - E_z \Big |^{n}_{i} \right ) + \dfrac{1}{\mu} \Delta t \psi_H \Big |^{n}_{i+1}
\end{equation}

    Recall that

\begin{equation}
\dfrac{1}{\epsilon} \dfrac{\Delta t}{\Delta x} = \dfrac{\eta_0}{\epsilon_r} S_c \qquad \mathrm{and} \qquad 
\dfrac{1}{\mu} \dfrac{\Delta t}{\Delta x} = \dfrac{1}{\mu_r \eta_0} S_c
\end{equation}

\textbackslash{}end\{equation\}

and we note that in our implementation \(\Delta t\) and \(\Delta x\),
and the Courant number are all set to unity.

The final update equations are now

\begin{equation}
H_y \Big |^{n+1/2}_{i+1/2} =  H_y \Big |^{n-1/2}_{i+1/2} + \dfrac{1}{\mu_r \eta_0} \left ( E_z \Big |^{n}_{i+1} - E_z \Big |^{n}_{i} \right ) + \dfrac{1}{\mu_r \eta_0} \psi_H \Big |^{n}_{i+1}
\end{equation}

\begin{equation}
E_z \Big |^{n+1}_{i} = E_z \Big |^{n}_{i} + \dfrac{\eta_0}{\epsilon_r} \left ( H_y \Big |^{n+1/2}_{i+1/2} - H_y \Big |^{n+1/2}_{i-1/2} \right ) + \dfrac{\eta_0}{\epsilon_r} \psi_E \Big |^{n+1/2}_{i+1/2}
\end{equation}

    \subsection*{Python implementation}\label{python-implementation}

First, we begin with importing the libraries for plotting and for the
mathematical functions used in the simulation.

    \begin{Verbatim}[commandchars=\\\{\}]
{\color{incolor}In [{\color{incolor}13}]:} \PY{o}{\PYZpc{}}\PY{k}{matplotlib} inline
         \PY{k+kn}{from} \PY{n+nn}{matplotlib} \PY{k}{import} \PY{n}{pyplot} \PY{k}{as} \PY{n}{plt}
         \PY{k+kn}{import} \PY{n+nn}{numpy} \PY{k}{as} \PY{n+nn}{np}
         \PY{k+kn}{import} \PY{n+nn}{math} \PY{k}{as} \PY{n+nn}{m}
         
         \PY{n}{fignum} \PY{o}{=} \PY{l+m+mi}{0}
\end{Verbatim}

    We also define the size of the domain in both space and time, as well as
a few other parameters for the source. Here, we also initialize the
electric and magnetic fields to be zero, and the position array.

    \begin{Verbatim}[commandchars=\\\{\}]
{\color{incolor}In [{\color{incolor}14}]:} \PY{n}{epsilon} \PY{o}{=} \PY{l+m+mi}{25}                      \PY{c+c1}{\PYZsh{}permittivity}
         \PY{n}{imp0} \PY{o}{=} \PY{l+m+mf}{337.0}                      \PY{c+c1}{\PYZsh{}impedance}
         
         \PY{n}{nx} \PY{o}{=} \PY{l+m+mi}{500}
         
         \PY{n}{srcori} \PY{o}{=} \PY{n+nb}{int}\PY{p}{(}\PY{n}{nx}\PY{o}{/}\PY{l+m+mi}{2}\PY{p}{)}                \PY{c+c1}{\PYZsh{}source is at center}
         \PY{n}{srcwid} \PY{o}{=} \PY{l+m+mf}{30.0}\PY{o}{*}\PY{n}{np}\PY{o}{.}\PY{n}{sqrt}\PY{p}{(}\PY{n}{epsilon}\PY{p}{)}    \PY{c+c1}{\PYZsh{}source width}
         \PY{n}{srcdel} \PY{o}{=} \PY{l+m+mi}{10}\PY{o}{*}\PY{n}{srcwid}                \PY{c+c1}{\PYZsh{}source delay}
         
         \PY{n}{nt} \PY{o}{=} \PY{n+nb}{int}\PY{p}{(}\PY{p}{(}\PY{n}{nx}\PY{o}{+}\PY{n}{srcdel}\PY{p}{)}\PY{o}{*}\PY{n}{np}\PY{o}{.}\PY{n}{sqrt}\PY{p}{(}\PY{n}{epsilon}\PY{p}{)}\PY{p}{)}
         
         \PY{n}{ez} \PY{o}{=} \PY{n}{np}\PY{o}{.}\PY{n}{zeros}\PY{p}{(}\PY{n}{nx}\PY{p}{)}
         \PY{n}{hy} \PY{o}{=} \PY{n}{np}\PY{o}{.}\PY{n}{zeros}\PY{p}{(}\PY{n}{nx}\PY{p}{)}
         \PY{n}{x} \PY{o}{=} \PY{n}{np}\PY{o}{.}\PY{n}{arange}\PY{p}{(}\PY{l+m+mi}{0}\PY{p}{,}\PY{n}{nx}\PY{o}{\PYZhy{}}\PY{l+m+mi}{1}\PY{p}{,}\PY{l+m+mi}{1}\PY{p}{)}
\end{Verbatim}

    We define variables for the reflection factor \(R(0)\), the polynomial
grading \(m\), the thickness of the PML \(\Delta PML\), and the maximum
conductivity \(\sigma_{max}\)

    \begin{Verbatim}[commandchars=\\\{\}]
{\color{incolor}In [{\color{incolor}15}]:} \PY{n}{R0} \PY{o}{=} \PY{l+m+mi}{1}\PY{n}{e}\PY{o}{\PYZhy{}}\PY{l+m+mi}{6}          \PY{c+c1}{\PYZsh{} reflection factor}
         \PY{n}{gra} \PY{o}{=} \PY{l+m+mi}{4}             \PY{c+c1}{\PYZsh{} order of polynomial grading}
         \PY{n}{dpml} \PY{o}{=} \PY{l+m+mi}{10}           \PY{c+c1}{\PYZsh{} number of PML cells}
         
         \PY{n}{smax} \PY{o}{=} \PY{o}{\PYZhy{}}\PY{p}{(}\PY{p}{(}\PY{n}{gra}\PY{o}{+}\PY{l+m+mi}{1}\PY{p}{)}\PY{o}{*}\PY{n}{m}\PY{o}{.}\PY{n}{log}\PY{p}{(}\PY{n}{R0}\PY{p}{)}\PY{p}{)}\PY{o}{/}\PY{p}{(}\PY{l+m+mi}{2}\PY{o}{*}\PY{n}{imp0}\PY{o}{*}\PY{n}{dpml}\PY{p}{)}
\end{Verbatim}

    Defining the polynomial grading at the boundaries.

Here we define the arrays \texttt{es} and \texttt{hs} for the complex
phase \(\psi\) for the electric and magnetic fields. Note that the are
only defined from 0 to \(\Delta PML\) on the left side, and from \(nx\)
to \(nx-\Delta PML\) on the other end; for anywhere else in the array,
\texttt{es} and \texttt{hs} are zero.

    \begin{Verbatim}[commandchars=\\\{\}]
{\color{incolor}In [{\color{incolor}16}]:} \PY{n}{es} \PY{o}{=} \PY{n}{np}\PY{o}{.}\PY{n}{zeros}\PY{p}{(}\PY{n}{nx}\PY{p}{)}
         \PY{n}{hs} \PY{o}{=} \PY{n}{np}\PY{o}{.}\PY{n}{zeros}\PY{p}{(}\PY{n}{nx}\PY{p}{)}
         
         \PY{c+c1}{\PYZsh{}polynomial gradng of the conductivity at the boundaries}
         \PY{k}{for} \PY{n}{i} \PY{o+ow}{in} \PY{n+nb}{range}\PY{p}{(}\PY{n}{dpml}\PY{p}{)}\PY{p}{:}
             \PY{c+c1}{\PYZsh{}for the left side of the PML}
             \PY{n}{es}\PY{p}{[}\PY{n}{i}\PY{o}{+}\PY{l+m+mi}{1}\PY{p}{]} \PY{o}{=} \PY{n}{smax}\PY{o}{*}\PY{p}{(}\PY{p}{(}\PY{n}{dpml}\PY{o}{\PYZhy{}}\PY{n}{i}\PY{o}{\PYZhy{}}\PY{l+m+mf}{0.5}\PY{p}{)}\PY{o}{/}\PY{n}{dpml}\PY{p}{)}\PY{o}{*}\PY{o}{*}\PY{n}{gra}
             \PY{n}{hs}\PY{p}{[}\PY{n}{i}\PY{p}{]} \PY{o}{=} \PY{n}{smax}\PY{o}{*}\PY{p}{(}\PY{p}{(}\PY{n}{dpml}\PY{o}{\PYZhy{}}\PY{n}{i}\PY{p}{)}\PY{o}{/}\PY{n}{dpml}\PY{p}{)}\PY{o}{*}\PY{o}{*}\PY{n}{gra}  
             
             \PY{c+c1}{\PYZsh{}for the right side of the PML}
             \PY{n}{es}\PY{p}{[}\PY{n}{nx}\PY{o}{\PYZhy{}}\PY{n}{i}\PY{o}{\PYZhy{}}\PY{l+m+mi}{1}\PY{p}{]} \PY{o}{=} \PY{n}{smax}\PY{o}{*}\PY{p}{(}\PY{p}{(}\PY{n}{dpml}\PY{o}{\PYZhy{}}\PY{n}{i}\PY{o}{\PYZhy{}}\PY{l+m+mf}{0.5}\PY{p}{)}\PY{o}{/}\PY{n}{dpml}\PY{p}{)}\PY{o}{*}\PY{o}{*}\PY{n}{gra} 
             \PY{n}{hs}\PY{p}{[}\PY{n}{nx}\PY{o}{\PYZhy{}}\PY{n}{i}\PY{o}{\PYZhy{}}\PY{l+m+mi}{1}\PY{p}{]} \PY{o}{=} \PY{n}{smax}\PY{o}{*}\PY{p}{(}\PY{p}{(}\PY{n}{dpml}\PY{o}{\PYZhy{}}\PY{n}{i}\PY{p}{)}\PY{o}{/}\PY{n}{dpml}\PY{p}{)}\PY{o}{*}\PY{o}{*}\PY{n}{gra}
\end{Verbatim}

    We define constants \(a\) and \(b\) for both electric and magnetic
fields, and we initialize the array for the phase \(\psi\).

    \begin{Verbatim}[commandchars=\\\{\}]
{\color{incolor}In [{\color{incolor}17}]:} \PY{n}{ea} \PY{o}{=} \PY{n}{np}\PY{o}{.}\PY{n}{exp}\PY{p}{(}\PY{o}{\PYZhy{}}\PY{n}{es}\PY{o}{*}\PY{n}{imp0}\PY{p}{)}\PY{o}{\PYZhy{}}\PY{l+m+mi}{1}
         \PY{n}{eb} \PY{o}{=} \PY{n}{np}\PY{o}{.}\PY{n}{exp}\PY{p}{(}\PY{o}{\PYZhy{}}\PY{n}{es}\PY{o}{*}\PY{n}{imp0}\PY{p}{)}
         
         \PY{n}{ha} \PY{o}{=} \PY{n}{np}\PY{o}{.}\PY{n}{exp}\PY{p}{(}\PY{o}{\PYZhy{}}\PY{n}{hs}\PY{o}{*}\PY{n}{imp0}\PY{p}{)}\PY{o}{\PYZhy{}}\PY{l+m+mi}{1}
         \PY{n}{hb} \PY{o}{=} \PY{n}{np}\PY{o}{.}\PY{n}{exp}\PY{p}{(}\PY{o}{\PYZhy{}}\PY{n}{hs}\PY{o}{*}\PY{n}{imp0}\PY{p}{)}
         
         \PY{n}{psihy} \PY{o}{=} \PY{n}{np}\PY{o}{.}\PY{n}{zeros}\PY{p}{(}\PY{n}{nx}\PY{p}{)}
         \PY{n}{psiez} \PY{o}{=} \PY{n}{np}\PY{o}{.}\PY{n}{zeros}\PY{p}{(}\PY{n}{nx}\PY{p}{)}
\end{Verbatim}

    The loop for generating the fields. Here, we added the added phase for
the PML \texttt{psihy} and \texttt{psiez}. These phases change the
fields at the PML domains.

Notice that after each reflection, the field amplitude is drastically
reduced, and the reflection is reduced as $\Delta PML$ is increased.

    \begin{Verbatim}[commandchars=\\\{\}]
{\color{incolor}In [{\color{incolor}18}]:} \PY{k}{for} \PY{n}{dt} \PY{o+ow}{in} \PY{n+nb}{range}\PY{p}{(}\PY{l+m+mi}{0}\PY{p}{,}\PY{n}{nt}\PY{p}{)}\PY{p}{:}
             \PY{n}{psihy}\PY{p}{[}\PY{n}{x}\PY{p}{]} \PY{o}{=} \PY{n}{hb}\PY{p}{[}\PY{n}{x}\PY{p}{]}\PY{o}{*}\PY{n}{psihy}\PY{p}{[}\PY{n}{x}\PY{p}{]} \PY{o}{+} \PY{n}{ha}\PY{p}{[}\PY{n}{x}\PY{p}{]}\PY{o}{*}\PY{p}{(}\PY{n}{ez}\PY{p}{[}\PY{n}{x}\PY{o}{+}\PY{l+m+mi}{1}\PY{p}{]} \PY{o}{\PYZhy{}} \PY{n}{ez}\PY{p}{[}\PY{n}{x}\PY{p}{]}\PY{p}{)}
             \PY{n}{hy}\PY{p}{[}\PY{n}{x}\PY{p}{]} \PY{o}{=} \PY{n}{hy}\PY{p}{[}\PY{n}{x}\PY{p}{]} \PY{o}{+} \PY{p}{(}\PY{n}{ez}\PY{p}{[}\PY{n}{x}\PY{o}{+}\PY{l+m+mi}{1}\PY{p}{]} \PY{o}{\PYZhy{}} \PY{n}{ez}\PY{p}{[}\PY{n}{x}\PY{p}{]}\PY{p}{)}\PY{o}{/}\PY{n}{imp0} \PY{o}{+} \PY{n}{psihy}\PY{p}{[}\PY{n}{x}\PY{p}{]}\PY{o}{/}\PY{n}{imp0}
             
             \PY{n}{psiez}\PY{p}{[}\PY{n}{x}\PY{o}{+}\PY{l+m+mi}{1}\PY{p}{]} \PY{o}{=} \PY{n}{eb}\PY{p}{[}\PY{n}{x}\PY{o}{+}\PY{l+m+mi}{1}\PY{p}{]}\PY{o}{*}\PY{n}{psiez}\PY{p}{[}\PY{n}{x}\PY{o}{+}\PY{l+m+mi}{1}\PY{p}{]} \PY{o}{+} \PY{n}{ea}\PY{p}{[}\PY{n}{x}\PY{o}{+}\PY{l+m+mi}{1}\PY{p}{]}\PY{o}{*}\PY{p}{(}\PY{n}{hy}\PY{p}{[}\PY{n}{x}\PY{o}{+}\PY{l+m+mi}{1}\PY{p}{]}\PY{o}{\PYZhy{}}\PY{n}{hy}\PY{p}{[}\PY{n}{x}\PY{p}{]}\PY{p}{)}
             \PY{n}{ez}\PY{p}{[}\PY{n}{x}\PY{o}{+}\PY{l+m+mi}{1}\PY{p}{]} \PY{o}{=} \PY{n}{ez}\PY{p}{[}\PY{n}{x}\PY{o}{+}\PY{l+m+mi}{1}\PY{p}{]} \PY{o}{+} \PY{p}{(}\PY{n}{hy}\PY{p}{[}\PY{n}{x}\PY{o}{+}\PY{l+m+mi}{1}\PY{p}{]}\PY{o}{\PYZhy{}}\PY{n}{hy}\PY{p}{[}\PY{n}{x}\PY{p}{]}\PY{p}{)}\PY{o}{*}\PY{n}{imp0}\PY{o}{/}\PY{n}{epsilon} \PY{o}{+} \PY{n}{psiez}\PY{p}{[}\PY{n}{x}\PY{o}{+}\PY{l+m+mi}{1}\PY{p}{]}\PY{o}{*}\PY{n}{imp0}\PY{o}{/}\PY{n}{epsilon}
         
             \PY{n}{ez}\PY{p}{[}\PY{n}{srcori}\PY{p}{]} \PY{o}{+}\PY{o}{=} \PY{n}{m}\PY{o}{.}\PY{n}{exp}\PY{p}{(}\PY{o}{\PYZhy{}}\PY{p}{(}\PY{p}{(}\PY{n}{dt}\PY{o}{\PYZhy{}}\PY{n}{srcdel}\PY{p}{)}\PY{o}{*}\PY{p}{(}\PY{n}{dt}\PY{o}{\PYZhy{}}\PY{n}{srcdel}\PY{p}{)}\PY{p}{)}\PY{o}{/}\PY{p}{(}\PY{n}{srcwid}\PY{o}{*}\PY{n}{srcwid}\PY{p}{)}\PY{p}{)}
         
             \PY{n}{plt}\PY{o}{.}\PY{n}{hold}\PY{p}{(}\PY{k+kc}{True}\PY{p}{)}
             \PY{k}{if} \PY{p}{(}\PY{n}{dt} \PY{o}{\PYZpc{}} \PY{l+m+mi}{1000} \PY{o}{==} \PY{l+m+mi}{0} \PY{o+ow}{and} \PY{n}{dt} \PY{o}{\PYZlt{}} \PY{l+m+mi}{7000} \PY{p}{)}\PY{p}{:}
                 \PY{n}{fignum} \PY{o}{=} \PY{n}{fignum} \PY{o}{+} \PY{l+m+mi}{1}
                 \PY{n}{plt}\PY{o}{.}\PY{n}{figure}\PY{p}{(}\PY{n}{fignum}\PY{p}{)}
                 \PY{n}{plt}\PY{o}{.}\PY{n}{title}\PY{p}{(}\PY{l+s+s2}{\PYZdq{}}\PY{l+s+s2}{Field at t = }\PY{l+s+s2}{\PYZdq{}} \PY{o}{+} \PY{n+nb}{str}\PY{p}{(}\PY{n}{dt}\PY{p}{)}\PY{p}{)}
                 \PY{n}{plt}\PY{o}{.}\PY{n}{ylabel}\PY{p}{(}\PY{l+s+s2}{\PYZdq{}}\PY{l+s+s2}{Field Amplitude}\PY{l+s+s2}{\PYZdq{}}\PY{p}{)}
                 \PY{n}{plt}\PY{o}{.}\PY{n}{xlabel}\PY{p}{(}\PY{l+s+s2}{\PYZdq{}}\PY{l+s+s2}{Position}\PY{l+s+s2}{\PYZdq{}}\PY{p}{)}
                 \PY{n}{plt}\PY{o}{.}\PY{n}{plot}\PY{p}{(}\PY{n}{ez}\PY{p}{,} \PY{n}{label}\PY{o}{=}\PY{l+s+s2}{\PYZdq{}}\PY{l+s+s2}{E\PYZhy{}Field}\PY{l+s+s2}{\PYZdq{}}\PY{p}{)}
                 \PY{n}{plt}\PY{o}{.}\PY{n}{plot}\PY{p}{(}\PY{n}{hy}\PY{o}{*}\PY{n}{imp0}\PY{p}{,} \PY{n}{label}\PY{o}{=}\PY{l+s+s2}{\PYZdq{}}\PY{l+s+s2}{H\PYZhy{}Field}\PY{l+s+s2}{\PYZdq{}}\PY{p}{)}
                 \PY{n}{plt}\PY{o}{.}\PY{n}{legend}\PY{p}{(}\PY{p}{)}
\end{Verbatim}

\subsection*{PML simulation results for $\Delta PML = 5$}

    \begin{center}
    \adjustimage{max size={0.5\linewidth}{0.5\paperheight}}{task2_files/5/task2_19_0.png}
    \end{center}
    { \hspace*{\fill} \\}
    
    \begin{center}
    \adjustimage{max size={0.5\linewidth}{0.5\paperheight}}{task2_files/5/task2_19_1.png}
    \end{center}
    { \hspace*{\fill} \\}
    
    \begin{center}
    \adjustimage{max size={0.5\linewidth}{0.5\paperheight}}{task2_files/5/task2_19_2.png}
    \end{center}
    { \hspace*{\fill} \\}
    
    \begin{center}
    \adjustimage{max size={0.5\linewidth}{0.5\paperheight}}{task2_files/5/task2_19_3.png}
    \end{center}
    { \hspace*{\fill} \\}
    
    \begin{center}
    \adjustimage{max size={0.5\linewidth}{0.5\paperheight}}{task2_files/5/task2_19_4.png}
    \end{center}
    { \hspace*{\fill} \\}
    
    \begin{center}
    \adjustimage{max size={0.5\linewidth}{0.5\paperheight}}{task2_files/5/task2_19_5.png}
    \end{center}
    { \hspace*{\fill} \\}
    
    \begin{center}
    \adjustimage{max size={0.5\linewidth}{0.5\paperheight}}{task2_files/5/task2_19_6.png}
    \end{center}
    { \hspace*{\fill} \\}
    

\subsection*{PML simulation results for $\Delta PML = 10$}

    \begin{center}
    \adjustimage{max size={0.5\linewidth}{0.5\paperheight}}{task2_files/10/task2_19_0.png}
    \end{center}
    { \hspace*{\fill} \\}
    
    \begin{center}
    \adjustimage{max size={0.5\linewidth}{0.5\paperheight}}{task2_files/10/task2_19_1.png}
    \end{center}
    { \hspace*{\fill} \\}
    
    \begin{center}
    \adjustimage{max size={0.5\linewidth}{0.5\paperheight}}{task2_files/10/task2_19_2.png}
    \end{center}
    { \hspace*{\fill} \\}
    
    \begin{center}
    \adjustimage{max size={0.5\linewidth}{0.5\paperheight}}{task2_files/10/task2_19_3.png}
    \end{center}
    { \hspace*{\fill} \\}
    
    \begin{center}
    \adjustimage{max size={0.5\linewidth}{0.5\paperheight}}{task2_files/10/task2_19_4.png}
    \end{center}
    { \hspace*{\fill} \\}
    
    \begin{center}
    \adjustimage{max size={0.5\linewidth}{0.5\paperheight}}{task2_files/10/task2_19_5.png}
    \end{center}
    { \hspace*{\fill} \\}
    
    \begin{center}
    \adjustimage{max size={0.5\linewidth}{0.5\paperheight}}{task2_files/10/task2_19_6.png}
    \end{center}
    { \hspace*{\fill} \\}


\subsection*{PML simulation results for $\Delta PML = 20$}

    \begin{center}
    \adjustimage{max size={0.5\linewidth}{0.5\paperheight}}{task2_files/20/task2_19_0.png}
    \end{center}
    { \hspace*{\fill} \\}
    
    \begin{center}
    \adjustimage{max size={0.5\linewidth}{0.5\paperheight}}{task2_files/20/task2_19_1.png}
    \end{center}
    { \hspace*{\fill} \\}
    
    \begin{center}
    \adjustimage{max size={0.5\linewidth}{0.5\paperheight}}{task2_files/20/task2_19_2.png}
    \end{center}
    { \hspace*{\fill} \\}
    
    \begin{center}
    \adjustimage{max size={0.5\linewidth}{0.5\paperheight}}{task2_files/20/task2_19_3.png}
    \end{center}
    { \hspace*{\fill} \\}
    
    \begin{center}
    \adjustimage{max size={0.5\linewidth}{0.5\paperheight}}{task2_files/20/task2_19_4.png}
    \end{center}
    { \hspace*{\fill} \\}
    
    \begin{center}
    \adjustimage{max size={0.5\linewidth}{0.5\paperheight}}{task2_files/20/task2_19_5.png}
    \end{center}
    { \hspace*{\fill} \\}
    
    \begin{center}
    \adjustimage{max size={0.5\linewidth}{0.5\paperheight}}{task2_files/20/task2_19_6.png}
    \end{center}
    { \hspace*{\fill} \\}

    \subsection*{References}\label{references}

Inan, U. S. \& Marshall, R. A. \emph{Numerical electromagnetics: the
FDTD method.} (Cambridge University Press, 2011).

Roden, J. A. \& Gedney, S. D. \emph{Convolution PML (CPML): An efficient
FDTD implementation of the CFS-PML for arbitrary media.} Microwave and
Optical Technology Letters 27, 334--339 (2000).

Schneider, J. B. \emph{Understanding the finite-difference time-domain
method.} School of electrical engineering and computer science
Washington State University.--URL: http://www. Eecs. Wsu. Edu/∼
schneidj/ufdtd/(request data: 29.11. 2012) (2010).


    % Add a bibliography block to the postdoc
    
    
    
    \end{document}
