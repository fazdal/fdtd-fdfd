
% Default to the notebook output style

    


% Inherit from the specified cell style.




    
\documentclass{article}

    
    
    \usepackage{graphicx} % Used to insert images
    \usepackage{adjustbox} % Used to constrain images to a maximum size 
    \usepackage{color} % Allow colors to be defined
    \usepackage{enumerate} % Needed for markdown enumerations to work
    \usepackage{geometry} % Used to adjust the document margins
    \usepackage{amsmath} % Equations
    \usepackage{amssymb} % Equations
    \usepackage{eurosym} % defines \euro
    \usepackage[mathletters]{ucs} % Extended unicode (utf-8) support
    \usepackage[utf8x]{inputenc} % Allow utf-8 characters in the tex document
    \usepackage{fancyvrb} % verbatim replacement that allows latex
    \usepackage{grffile} % extends the file name processing of package graphics 
                         % to support a larger range 
    % The hyperref package gives us a pdf with properly built
    % internal navigation ('pdf bookmarks' for the table of contents,
    % internal cross-reference links, web links for URLs, etc.)
    \usepackage{hyperref}
    \usepackage{longtable} % longtable support required by pandoc >1.10
    \usepackage{booktabs}  % table support for pandoc > 1.12.2
    \usepackage{ulem} % ulem is needed to support strikethroughs (\sout)
    

    
    
    \definecolor{orange}{cmyk}{0,0.4,0.8,0.2}
    \definecolor{darkorange}{rgb}{.71,0.21,0.01}
    \definecolor{darkgreen}{rgb}{.12,.54,.11}
    \definecolor{myteal}{rgb}{.26, .44, .56}
    \definecolor{gray}{gray}{0.45}
    \definecolor{lightgray}{gray}{.95}
    \definecolor{mediumgray}{gray}{.8}
    \definecolor{inputbackground}{rgb}{.95, .95, .85}
    \definecolor{outputbackground}{rgb}{.95, .95, .95}
    \definecolor{traceback}{rgb}{1, .95, .95}
    % ansi colors
    \definecolor{red}{rgb}{.6,0,0}
    \definecolor{green}{rgb}{0,.65,0}
    \definecolor{brown}{rgb}{0.6,0.6,0}
    \definecolor{blue}{rgb}{0,.145,.698}
    \definecolor{purple}{rgb}{.698,.145,.698}
    \definecolor{cyan}{rgb}{0,.698,.698}
    \definecolor{lightgray}{gray}{0.5}
    
    % bright ansi colors
    \definecolor{darkgray}{gray}{0.25}
    \definecolor{lightred}{rgb}{1.0,0.39,0.28}
    \definecolor{lightgreen}{rgb}{0.48,0.99,0.0}
    \definecolor{lightblue}{rgb}{0.53,0.81,0.92}
    \definecolor{lightpurple}{rgb}{0.87,0.63,0.87}
    \definecolor{lightcyan}{rgb}{0.5,1.0,0.83}
    
    % commands and environments needed by pandoc snippets
    % extracted from the output of `pandoc -s`
    \providecommand{\tightlist}{%
      \setlength{\itemsep}{0pt}\setlength{\parskip}{0pt}}
    \DefineVerbatimEnvironment{Highlighting}{Verbatim}{commandchars=\\\{\}}
    % Add ',fontsize=\small' for more characters per line
    \newenvironment{Shaded}{}{}
    \newcommand{\KeywordTok}[1]{\textcolor[rgb]{0.00,0.44,0.13}{\textbf{{#1}}}}
    \newcommand{\DataTypeTok}[1]{\textcolor[rgb]{0.56,0.13,0.00}{{#1}}}
    \newcommand{\DecValTok}[1]{\textcolor[rgb]{0.25,0.63,0.44}{{#1}}}
    \newcommand{\BaseNTok}[1]{\textcolor[rgb]{0.25,0.63,0.44}{{#1}}}
    \newcommand{\FloatTok}[1]{\textcolor[rgb]{0.25,0.63,0.44}{{#1}}}
    \newcommand{\CharTok}[1]{\textcolor[rgb]{0.25,0.44,0.63}{{#1}}}
    \newcommand{\StringTok}[1]{\textcolor[rgb]{0.25,0.44,0.63}{{#1}}}
    \newcommand{\CommentTok}[1]{\textcolor[rgb]{0.38,0.63,0.69}{\textit{{#1}}}}
    \newcommand{\OtherTok}[1]{\textcolor[rgb]{0.00,0.44,0.13}{{#1}}}
    \newcommand{\AlertTok}[1]{\textcolor[rgb]{1.00,0.00,0.00}{\textbf{{#1}}}}
    \newcommand{\FunctionTok}[1]{\textcolor[rgb]{0.02,0.16,0.49}{{#1}}}
    \newcommand{\RegionMarkerTok}[1]{{#1}}
    \newcommand{\ErrorTok}[1]{\textcolor[rgb]{1.00,0.00,0.00}{\textbf{{#1}}}}
    \newcommand{\NormalTok}[1]{{#1}}
    
    % Additional commands for more recent versions of Pandoc
    \newcommand{\ConstantTok}[1]{\textcolor[rgb]{0.53,0.00,0.00}{{#1}}}
    \newcommand{\SpecialCharTok}[1]{\textcolor[rgb]{0.25,0.44,0.63}{{#1}}}
    \newcommand{\VerbatimStringTok}[1]{\textcolor[rgb]{0.25,0.44,0.63}{{#1}}}
    \newcommand{\SpecialStringTok}[1]{\textcolor[rgb]{0.73,0.40,0.53}{{#1}}}
    \newcommand{\ImportTok}[1]{{#1}}
    \newcommand{\DocumentationTok}[1]{\textcolor[rgb]{0.73,0.13,0.13}{\textit{{#1}}}}
    \newcommand{\AnnotationTok}[1]{\textcolor[rgb]{0.38,0.63,0.69}{\textbf{\textit{{#1}}}}}
    \newcommand{\CommentVarTok}[1]{\textcolor[rgb]{0.38,0.63,0.69}{\textbf{\textit{{#1}}}}}
    \newcommand{\VariableTok}[1]{\textcolor[rgb]{0.10,0.09,0.49}{{#1}}}
    \newcommand{\ControlFlowTok}[1]{\textcolor[rgb]{0.00,0.44,0.13}{\textbf{{#1}}}}
    \newcommand{\OperatorTok}[1]{\textcolor[rgb]{0.40,0.40,0.40}{{#1}}}
    \newcommand{\BuiltInTok}[1]{{#1}}
    \newcommand{\ExtensionTok}[1]{{#1}}
    \newcommand{\PreprocessorTok}[1]{\textcolor[rgb]{0.74,0.48,0.00}{{#1}}}
    \newcommand{\AttributeTok}[1]{\textcolor[rgb]{0.49,0.56,0.16}{{#1}}}
    \newcommand{\InformationTok}[1]{\textcolor[rgb]{0.38,0.63,0.69}{\textbf{\textit{{#1}}}}}
    \newcommand{\WarningTok}[1]{\textcolor[rgb]{0.38,0.63,0.69}{\textbf{\textit{{#1}}}}}
    
    
    % Define a nice break command that doesn't care if a line doesn't already
    % exist.
    \def\br{\hspace*{\fill} \\* }
    % Math Jax compatability definitions
    \def\gt{>}
    \def\lt{<}
    % Document parameters
    \title{task1}
    
    
    

    % Pygments definitions
    
\makeatletter
\def\PY@reset{\let\PY@it=\relax \let\PY@bf=\relax%
    \let\PY@ul=\relax \let\PY@tc=\relax%
    \let\PY@bc=\relax \let\PY@ff=\relax}
\def\PY@tok#1{\csname PY@tok@#1\endcsname}
\def\PY@toks#1+{\ifx\relax#1\empty\else%
    \PY@tok{#1}\expandafter\PY@toks\fi}
\def\PY@do#1{\PY@bc{\PY@tc{\PY@ul{%
    \PY@it{\PY@bf{\PY@ff{#1}}}}}}}
\def\PY#1#2{\PY@reset\PY@toks#1+\relax+\PY@do{#2}}

\expandafter\def\csname PY@tok@mo\endcsname{\def\PY@tc##1{\textcolor[rgb]{0.40,0.40,0.40}{##1}}}
\expandafter\def\csname PY@tok@w\endcsname{\def\PY@tc##1{\textcolor[rgb]{0.73,0.73,0.73}{##1}}}
\expandafter\def\csname PY@tok@s2\endcsname{\def\PY@tc##1{\textcolor[rgb]{0.73,0.13,0.13}{##1}}}
\expandafter\def\csname PY@tok@nd\endcsname{\def\PY@tc##1{\textcolor[rgb]{0.67,0.13,1.00}{##1}}}
\expandafter\def\csname PY@tok@sh\endcsname{\def\PY@tc##1{\textcolor[rgb]{0.73,0.13,0.13}{##1}}}
\expandafter\def\csname PY@tok@gp\endcsname{\let\PY@bf=\textbf\def\PY@tc##1{\textcolor[rgb]{0.00,0.00,0.50}{##1}}}
\expandafter\def\csname PY@tok@s\endcsname{\def\PY@tc##1{\textcolor[rgb]{0.73,0.13,0.13}{##1}}}
\expandafter\def\csname PY@tok@gr\endcsname{\def\PY@tc##1{\textcolor[rgb]{1.00,0.00,0.00}{##1}}}
\expandafter\def\csname PY@tok@gs\endcsname{\let\PY@bf=\textbf}
\expandafter\def\csname PY@tok@cpf\endcsname{\let\PY@it=\textit\def\PY@tc##1{\textcolor[rgb]{0.25,0.50,0.50}{##1}}}
\expandafter\def\csname PY@tok@c\endcsname{\let\PY@it=\textit\def\PY@tc##1{\textcolor[rgb]{0.25,0.50,0.50}{##1}}}
\expandafter\def\csname PY@tok@gi\endcsname{\def\PY@tc##1{\textcolor[rgb]{0.00,0.63,0.00}{##1}}}
\expandafter\def\csname PY@tok@m\endcsname{\def\PY@tc##1{\textcolor[rgb]{0.40,0.40,0.40}{##1}}}
\expandafter\def\csname PY@tok@vg\endcsname{\def\PY@tc##1{\textcolor[rgb]{0.10,0.09,0.49}{##1}}}
\expandafter\def\csname PY@tok@mi\endcsname{\def\PY@tc##1{\textcolor[rgb]{0.40,0.40,0.40}{##1}}}
\expandafter\def\csname PY@tok@err\endcsname{\def\PY@bc##1{\setlength{\fboxsep}{0pt}\fcolorbox[rgb]{1.00,0.00,0.00}{1,1,1}{\strut ##1}}}
\expandafter\def\csname PY@tok@go\endcsname{\def\PY@tc##1{\textcolor[rgb]{0.53,0.53,0.53}{##1}}}
\expandafter\def\csname PY@tok@cs\endcsname{\let\PY@it=\textit\def\PY@tc##1{\textcolor[rgb]{0.25,0.50,0.50}{##1}}}
\expandafter\def\csname PY@tok@vi\endcsname{\def\PY@tc##1{\textcolor[rgb]{0.10,0.09,0.49}{##1}}}
\expandafter\def\csname PY@tok@ow\endcsname{\let\PY@bf=\textbf\def\PY@tc##1{\textcolor[rgb]{0.67,0.13,1.00}{##1}}}
\expandafter\def\csname PY@tok@gu\endcsname{\let\PY@bf=\textbf\def\PY@tc##1{\textcolor[rgb]{0.50,0.00,0.50}{##1}}}
\expandafter\def\csname PY@tok@kn\endcsname{\let\PY@bf=\textbf\def\PY@tc##1{\textcolor[rgb]{0.00,0.50,0.00}{##1}}}
\expandafter\def\csname PY@tok@kp\endcsname{\def\PY@tc##1{\textcolor[rgb]{0.00,0.50,0.00}{##1}}}
\expandafter\def\csname PY@tok@nf\endcsname{\def\PY@tc##1{\textcolor[rgb]{0.00,0.00,1.00}{##1}}}
\expandafter\def\csname PY@tok@s1\endcsname{\def\PY@tc##1{\textcolor[rgb]{0.73,0.13,0.13}{##1}}}
\expandafter\def\csname PY@tok@kc\endcsname{\let\PY@bf=\textbf\def\PY@tc##1{\textcolor[rgb]{0.00,0.50,0.00}{##1}}}
\expandafter\def\csname PY@tok@nt\endcsname{\let\PY@bf=\textbf\def\PY@tc##1{\textcolor[rgb]{0.00,0.50,0.00}{##1}}}
\expandafter\def\csname PY@tok@ch\endcsname{\let\PY@it=\textit\def\PY@tc##1{\textcolor[rgb]{0.25,0.50,0.50}{##1}}}
\expandafter\def\csname PY@tok@gd\endcsname{\def\PY@tc##1{\textcolor[rgb]{0.63,0.00,0.00}{##1}}}
\expandafter\def\csname PY@tok@kd\endcsname{\let\PY@bf=\textbf\def\PY@tc##1{\textcolor[rgb]{0.00,0.50,0.00}{##1}}}
\expandafter\def\csname PY@tok@sb\endcsname{\def\PY@tc##1{\textcolor[rgb]{0.73,0.13,0.13}{##1}}}
\expandafter\def\csname PY@tok@bp\endcsname{\def\PY@tc##1{\textcolor[rgb]{0.00,0.50,0.00}{##1}}}
\expandafter\def\csname PY@tok@no\endcsname{\def\PY@tc##1{\textcolor[rgb]{0.53,0.00,0.00}{##1}}}
\expandafter\def\csname PY@tok@il\endcsname{\def\PY@tc##1{\textcolor[rgb]{0.40,0.40,0.40}{##1}}}
\expandafter\def\csname PY@tok@se\endcsname{\let\PY@bf=\textbf\def\PY@tc##1{\textcolor[rgb]{0.73,0.40,0.13}{##1}}}
\expandafter\def\csname PY@tok@cm\endcsname{\let\PY@it=\textit\def\PY@tc##1{\textcolor[rgb]{0.25,0.50,0.50}{##1}}}
\expandafter\def\csname PY@tok@k\endcsname{\let\PY@bf=\textbf\def\PY@tc##1{\textcolor[rgb]{0.00,0.50,0.00}{##1}}}
\expandafter\def\csname PY@tok@sr\endcsname{\def\PY@tc##1{\textcolor[rgb]{0.73,0.40,0.53}{##1}}}
\expandafter\def\csname PY@tok@o\endcsname{\def\PY@tc##1{\textcolor[rgb]{0.40,0.40,0.40}{##1}}}
\expandafter\def\csname PY@tok@gt\endcsname{\def\PY@tc##1{\textcolor[rgb]{0.00,0.27,0.87}{##1}}}
\expandafter\def\csname PY@tok@ge\endcsname{\let\PY@it=\textit}
\expandafter\def\csname PY@tok@vc\endcsname{\def\PY@tc##1{\textcolor[rgb]{0.10,0.09,0.49}{##1}}}
\expandafter\def\csname PY@tok@sd\endcsname{\let\PY@it=\textit\def\PY@tc##1{\textcolor[rgb]{0.73,0.13,0.13}{##1}}}
\expandafter\def\csname PY@tok@nv\endcsname{\def\PY@tc##1{\textcolor[rgb]{0.10,0.09,0.49}{##1}}}
\expandafter\def\csname PY@tok@si\endcsname{\let\PY@bf=\textbf\def\PY@tc##1{\textcolor[rgb]{0.73,0.40,0.53}{##1}}}
\expandafter\def\csname PY@tok@sx\endcsname{\def\PY@tc##1{\textcolor[rgb]{0.00,0.50,0.00}{##1}}}
\expandafter\def\csname PY@tok@na\endcsname{\def\PY@tc##1{\textcolor[rgb]{0.49,0.56,0.16}{##1}}}
\expandafter\def\csname PY@tok@ne\endcsname{\let\PY@bf=\textbf\def\PY@tc##1{\textcolor[rgb]{0.82,0.25,0.23}{##1}}}
\expandafter\def\csname PY@tok@nn\endcsname{\let\PY@bf=\textbf\def\PY@tc##1{\textcolor[rgb]{0.00,0.00,1.00}{##1}}}
\expandafter\def\csname PY@tok@nl\endcsname{\def\PY@tc##1{\textcolor[rgb]{0.63,0.63,0.00}{##1}}}
\expandafter\def\csname PY@tok@gh\endcsname{\let\PY@bf=\textbf\def\PY@tc##1{\textcolor[rgb]{0.00,0.00,0.50}{##1}}}
\expandafter\def\csname PY@tok@mb\endcsname{\def\PY@tc##1{\textcolor[rgb]{0.40,0.40,0.40}{##1}}}
\expandafter\def\csname PY@tok@c1\endcsname{\let\PY@it=\textit\def\PY@tc##1{\textcolor[rgb]{0.25,0.50,0.50}{##1}}}
\expandafter\def\csname PY@tok@sc\endcsname{\def\PY@tc##1{\textcolor[rgb]{0.73,0.13,0.13}{##1}}}
\expandafter\def\csname PY@tok@nc\endcsname{\let\PY@bf=\textbf\def\PY@tc##1{\textcolor[rgb]{0.00,0.00,1.00}{##1}}}
\expandafter\def\csname PY@tok@cp\endcsname{\def\PY@tc##1{\textcolor[rgb]{0.74,0.48,0.00}{##1}}}
\expandafter\def\csname PY@tok@mf\endcsname{\def\PY@tc##1{\textcolor[rgb]{0.40,0.40,0.40}{##1}}}
\expandafter\def\csname PY@tok@nb\endcsname{\def\PY@tc##1{\textcolor[rgb]{0.00,0.50,0.00}{##1}}}
\expandafter\def\csname PY@tok@mh\endcsname{\def\PY@tc##1{\textcolor[rgb]{0.40,0.40,0.40}{##1}}}
\expandafter\def\csname PY@tok@ni\endcsname{\let\PY@bf=\textbf\def\PY@tc##1{\textcolor[rgb]{0.60,0.60,0.60}{##1}}}
\expandafter\def\csname PY@tok@kr\endcsname{\let\PY@bf=\textbf\def\PY@tc##1{\textcolor[rgb]{0.00,0.50,0.00}{##1}}}
\expandafter\def\csname PY@tok@kt\endcsname{\def\PY@tc##1{\textcolor[rgb]{0.69,0.00,0.25}{##1}}}
\expandafter\def\csname PY@tok@ss\endcsname{\def\PY@tc##1{\textcolor[rgb]{0.10,0.09,0.49}{##1}}}

\def\PYZbs{\char`\\}
\def\PYZus{\char`\_}
\def\PYZob{\char`\{}
\def\PYZcb{\char`\}}
\def\PYZca{\char`\^}
\def\PYZam{\char`\&}
\def\PYZlt{\char`\<}
\def\PYZgt{\char`\>}
\def\PYZsh{\char`\#}
\def\PYZpc{\char`\%}
\def\PYZdl{\char`\$}
\def\PYZhy{\char`\-}
\def\PYZsq{\char`\'}
\def\PYZdq{\char`\"}
\def\PYZti{\char`\~}
% for compatibility with earlier versions
\def\PYZat{@}
\def\PYZlb{[}
\def\PYZrb{]}
\makeatother


    % Exact colors from NB
    \definecolor{incolor}{rgb}{0.0, 0.0, 0.5}
    \definecolor{outcolor}{rgb}{0.545, 0.0, 0.0}



    
    % Prevent overflowing lines due to hard-to-break entities
    \sloppy 
    % Setup hyperref package
    \hypersetup{
      breaklinks=true,  % so long urls are correctly broken across lines
      colorlinks=true,
      urlcolor=blue,
      linkcolor=darkorange,
      citecolor=darkgreen,
      }
    % Slightly bigger margins than the latex defaults
    
    \geometry{verbose,tmargin=1in,bmargin=1in,lmargin=1in,rmargin=1in}
    
    

    \begin{document}
    
    
    \maketitle
    
    

    
    \section{Task 1: Absorbing Boundary
Conditions}\label{task-1-absorbing-boundary-conditions}

\subsection{Background}\label{background}

Some calculations would require the background material to be extended
infinitely in at least one direction. However, given the limitations on
computer storage, we need to be able to create boundaries such that the
domain appears to be infinite.

Here, we consider the local boundary conditions designed by Engquist and
Majda in 1977. These were developed such that the amplitude of the
reflection coefficients were small, and that the BCs, along with the
associated differential equation, guarantee a well-posed initial
boundary value problem.

    \subsection{Derivation}\label{derivation}

For the 1-dimensional wave equation given by

\begin{equation}
\dfrac{\partial^2 U}{\partial x^2} = \dfrac{1}{c^2} \dfrac{\partial^2 U}{\partial t^2} 
\end{equation}

we can define an operator \(G\) where

\begin{equation}
G = \dfrac{\partial^2}{\partial x^2} - \dfrac{1}{c^2} \dfrac{\partial^2}{\partial t^2} 
\end{equation}

so that the wave equation can be written easily as

\begin{equation}
GU = 0
\end{equation}

We can separate \(G\) such that

\begin{equation}
G = G^+ G^-
\end{equation}

\begin{equation}
G^+ = \dfrac{\partial}{\partial x} + \dfrac{1}{c} \dfrac{\partial}{\partial t}
\end{equation}

\begin{equation}
G^- = \dfrac{\partial}{\partial x} - \dfrac{1}{c} \dfrac{\partial}{\partial t}
\end{equation}

At the left side of the boundary, where \(x=0\), the application of the
operator \(G^-\) to any wavefunction \(U\) exactly absorbs a plane wave
propagating toward the left boundary at any incident angle. Simularly,
the operator \(G^+\) acting on \(U\) absorbs a plane wave propagating
toward the right boundary. Hence we have

\begin{equation}
G^-U = 0 \qquad \longrightarrow \qquad \dfrac{\partial U}{\partial x} - \dfrac{1}{c} \dfrac{\partial U}{\partial t} = 0
\end{equation}

and at the right side, at \(x=d\),

\begin{equation}
G^+U = 0 \qquad \longrightarrow \qquad \dfrac{\partial U}{\partial x} + \dfrac{1}{c} \dfrac{\partial U}{\partial t} = 0
\end{equation}

We take the time derivative of both of the equations above

\begin{equation}
\dfrac{\partial^2 U}{\partial x \partial t} - \dfrac{1}{c} \dfrac{\partial^2 U}{\partial t^2} = 0 \qquad \mathrm{at} \, x = 0
\end{equation}

\begin{equation}
\dfrac{\partial^2 U}{\partial x \partial t} + \dfrac{1}{c} \dfrac{\partial^2 U}{\partial t^2} = 0 \qquad \mathrm{at} \, x = d
\end{equation}

in order to be able to use the following expressions for the partial
derivatives

\begin{align}
\left . \dfrac{\partial^2 W}{\partial x \partial t} \right |^n_{1/2} &= \dfrac{1}{2} \left [ \left( \dfrac{W \big|_{1}^{n+1} - W \big|_{0}^{n+1} }{\Delta t \Delta x} \right ) - \left( \dfrac{W \big|_{1}^{n-1} - W \big|_{0}^{n-1} }{\Delta t \Delta x} \right ) \right ] 
\\
&= \dfrac{1}{2} \left [ \left( \dfrac{W \big|_{i}^{n+1} - W \big|_{i-1}^{n+1} }{\Delta t \Delta x} \right ) - \left( \dfrac{W \big|_{i}^{n-1} - W \big|_{i-1}^{n-1} }{\Delta t \Delta x} \right ) \right ]
\end{align}

\begin{align}
\left . \dfrac{\partial^2 W}{\partial t^2} \right |^n_{1/2} &= \dfrac{1}{2} \left [ \left ( \dfrac{W \big|^{n+1}_{0} - 2 W \big|^{n}_{0} + W \big |^{n-1}_{0}}{(\Delta t)^2} \right ) + \left ( \dfrac{W \big|^{n+1}_{1} - 2 W \big|^{n}_{1} + W \big |^{n-1}_{1}}{(\Delta t)^2} \right ) \right ] 
\\
&= \dfrac{1}{2} \left [ \left ( \dfrac{W \big|^{n+1}_{i-1} - 2 W \big|^{n}_{i-1} + W \big |^{n-1}_{i-1}}{(\Delta t)^2} \right ) + \left ( \dfrac{W \big|^{n+1}_{i} - 2 W \big|^{n}_{i} + W \big |^{n-1}_{i}}{(\Delta t)^2} \right ) \right ]
\end{align}

    At \(x=0\), we want to obtain the function at the next time step given
the information from the previous cycle. Here, we are looking for
\(W \Big|^{n+1}_{0}\). Substituting the expressions for the partial
derivatives into the wave equation yields

\begin{equation}
W \big|^{n+1}_{0} = - W \big|_{1}^{n-1} + \left(\dfrac{c \Delta t - \Delta x}{c \Delta t + \Delta x}\right )\left (W \big|_{1}^{n+1}+W \big|_{0}^{n-1}\right )+\dfrac{2 \Delta x}{c \Delta t + \Delta x}\left( {W \big|^{n}_{0}}+ {W \big|^{n}_{1}}\right )
\end{equation}

In the code, we will initialize all values of \(W\) to be zero.

At one iteration, we have the value for \(W \Big|_{1}^{n+1}\). After we
use the above equation to obtain \(W \Big|^{n+1}_{0}\), the program will
then transfer the information in the variable \(W \Big|^{n-1}\) to
\(W \Big|^{n}\), and \(W \Big|^{n}\) to \(W \Big|^{n+1}\), so that the
following iteration can use these values.

    Similarly at the other side of the boundary, we want to obtain
\(W \Big|_{i}^{n+1}\).

\begin{equation}
W \big|_{i}^{n+1} = - W \big|_{i-1}^{n-1} + \left (W \big|_{i-1}^{n+1} + W \big |^{n-1}_{i} \right ) \left(\dfrac{c \Delta t - \Delta x}{c \Delta t + \Delta x}\right) + \dfrac{2 \Delta x}{c \Delta t + \Delta x}\left(W \big|^{n}_{i} + W \big|^{n}_{i-1} \right) 
\end{equation}

Similar to the left side of the boundary, we initialize all values of
\(W\) at zero. At a given iteration, we have \(W \Big |_{i-1}^{n+1}\).
After solving for \(W \Big|_{i}^{n+1}\), the program will cycle through
the variables in the same process as described in the left boundary.

    \section{Python Implementation}\label{python-implementation}

First, we start with importing several libraries for plotting,
mathematical functions, and numpy for the arrays. The variable fignum is
also defined; this is used to create several plots of the field.

    \begin{Verbatim}[commandchars=\\\{\}]
{\color{incolor}In [{\color{incolor}1}]:} \PY{o}{\PYZpc{}}\PY{k}{matplotlib} inline
        \PY{k+kn}{from} \PY{n+nn}{matplotlib} \PY{k}{import} \PY{n}{pyplot} \PY{k}{as} \PY{n}{plt}
        
        \PY{k+kn}{import} \PY{n+nn}{numpy} \PY{k}{as} \PY{n+nn}{np}
        \PY{k+kn}{import} \PY{n+nn}{math} \PY{k}{as} \PY{n+nn}{m}
        
        \PY{n}{fignum} \PY{o}{=} \PY{l+m+mi}{0}
\end{Verbatim}

    We then define constants for the size of the spatial domain \(nx\), the
wave impedance \(Z = \sqrt{\frac{\mu}{\epsilon}}\), and the speed of
light \(c = \frac{1}{\sqrt{\mu \epsilon}}\). We assume that the field is
propagating in vacuum, hence \(\epsilon = 1\) and \(\mu = 1\). The
spatial domain size is defined arbitrarily.

    \begin{Verbatim}[commandchars=\\\{\}]
{\color{incolor}In [{\color{incolor}2}]:} \PY{n}{nx} \PY{o}{=} \PY{l+m+mi}{300}
        \PY{n}{imp0} \PY{o}{=} \PY{l+m+mf}{337.0}
        \PY{n}{epsilon} \PY{o}{=} \PY{l+m+mi}{1}
        \PY{n}{c} \PY{o}{=} \PY{l+m+mi}{1}\PY{o}{/}\PY{n}{np}\PY{o}{.}\PY{n}{sqrt}\PY{p}{(}\PY{n}{epsilon}\PY{p}{)}
\end{Verbatim}

    For the Gaussian source, we introduce variables for its location, width,
and delay. The location of the source is placed at the center of the
domain, the width is defined arbitrarily, and the ratio of the delay
with respect to the source is taken from the previous task.

We also define the temporal domain at this point. The term nx+srcdel is
taken from the previous task, with the factor 1.5 added in order to view
the field as it interacts with the boundary conditions.

    \begin{Verbatim}[commandchars=\\\{\}]
{\color{incolor}In [{\color{incolor}3}]:} \PY{n}{srcori} \PY{o}{=} \PY{n+nb}{int}\PY{p}{(}\PY{n}{nx}\PY{o}{/}\PY{l+m+mi}{2}\PY{p}{)}              \PY{c+c1}{\PYZsh{}source origin}
        \PY{n}{srcwid} \PY{o}{=} \PY{l+m+mf}{30.0}\PY{o}{*}\PY{n}{np}\PY{o}{.}\PY{n}{sqrt}\PY{p}{(}\PY{n}{epsilon}\PY{p}{)}  \PY{c+c1}{\PYZsh{}source width}
        \PY{n}{srcdel} \PY{o}{=} \PY{l+m+mi}{10}\PY{o}{*}\PY{n}{srcwid}              \PY{c+c1}{\PYZsh{}source delay}
        \PY{n}{nt} \PY{o}{=} \PY{n+nb}{int}\PY{p}{(}\PY{l+m+mf}{1.5}\PY{o}{*}\PY{n}{nx}\PY{o}{+}\PY{n}{srcdel}\PY{p}{)}
\end{Verbatim}

    Now, we create arrays for the \(E_z\) and \(H_y\) components of the
field, along with an array \(x\) for the spatial coordinates. Both
fields are initialized at zero.

    \begin{Verbatim}[commandchars=\\\{\}]
{\color{incolor}In [{\color{incolor}4}]:} \PY{n}{ez} \PY{o}{=} \PY{n}{np}\PY{o}{.}\PY{n}{zeros}\PY{p}{(}\PY{n}{nx}\PY{p}{)}
        \PY{n}{hy} \PY{o}{=} \PY{n}{np}\PY{o}{.}\PY{n}{zeros}\PY{p}{(}\PY{n}{nx}\PY{p}{)}
        \PY{n}{x} \PY{o}{=} \PY{n}{np}\PY{o}{.}\PY{n}{arange}\PY{p}{(}\PY{l+m+mi}{0}\PY{p}{,}\PY{n}{nx}\PY{o}{\PYZhy{}}\PY{l+m+mi}{1}\PY{p}{,}\PY{l+m+mi}{1}\PY{p}{)}
\end{Verbatim}

    Note that we defined the domain as integer steps, not as a real spatial
interval divided into smaller intervals. Hence, the step \(dx\) and
\(dt\) are already predefined as unity. We define the constants \(a\)
and \(b\) such that

\begin{equation}
a = \dfrac{c \Delta t - \Delta x}{c \Delta t + \Delta x}
\end{equation}\begin{equation}
b = \dfrac{2 \Delta x}{c \Delta t + \Delta x}
\end{equation}

and we assign these to variables

    \begin{Verbatim}[commandchars=\\\{\}]
{\color{incolor}In [{\color{incolor}5}]:} \PY{n}{a} \PY{o}{=} \PY{p}{(}\PY{n}{c}\PY{o}{\PYZhy{}}\PY{l+m+mi}{1}\PY{p}{)}\PY{o}{/}\PY{p}{(}\PY{n}{c}\PY{o}{+}\PY{l+m+mi}{1}\PY{p}{)}
        \PY{n}{b} \PY{o}{=} \PY{l+m+mi}{2}\PY{o}{/}\PY{p}{(}\PY{n}{c} \PY{o}{+} \PY{l+m+mi}{1}\PY{p}{)}
\end{Verbatim}

    We also define variables for the boundary conditions, for both sides of
the E-field and for both sides of the H-field. At the left, we have

\begin{align}
hwnp10 \rightarrow H \Big |^{n+1} _{0}  &\qquad ewnp10 \rightarrow E |^{n+1} _{0}
\\
hwnm11 \rightarrow H \Big |^{n-1} _{0}  &\qquad ewnm11 \rightarrow E |^{n-1} _{0}
\\
hwnp11 \rightarrow H \Big |^{n+1} _{1}  &\qquad ewnp11 \rightarrow E |^{n+1} _{1}
\\
hwnm10 \rightarrow H \Big |^{n-1} _{0} &\qquad ewnm10 \rightarrow E |^{n-1} _{0}
\\
hwn0 \rightarrow H \Big |^{n} _{0} &\qquad ewn0  \rightarrow E |^{n} _{0}
\\
hwn1 \rightarrow H \Big |^{n} _{1} &\qquad ewn1 \rightarrow E |^{n} _{1}
\end{align}

And at the right,

\begin{align}
hwnp1im1 \rightarrow H \Big |^{n+1} _{i-1}  &\qquad ewnp1im1 \rightarrow E |^{n+1} _{i-1}
\\
hwnm1i \rightarrow H \Big |^{n-1} _{i-1}  &\qquad ewnm1i \rightarrow E |^{n-1} _{i-1}
\\
hwnp1i \rightarrow H \Big |^{n+1} _{i}  &\qquad ewnp1i \rightarrow E |^{n+1} _{i}
\\
hwnm1im1 \rightarrow H \Big |^{n-1} _{i-1} &\qquad ewnm1im1 \rightarrow E |^{n-1} _{i-1}
\\
hwnim1 \rightarrow H \Big |^{n} _{i-1} &\qquad ewnim1  \rightarrow E |^{n} _{i-1}
\\
hwni \rightarrow H \Big |^{n} _{i} &\qquad ewni \rightarrow E |^{n} _{i}
\end{align}

All of these variables are initialized to zero as the field is initially
zero.

    \begin{Verbatim}[commandchars=\\\{\}]
{\color{incolor}In [{\color{incolor}6}]:} \PY{n}{hwnp10}\PY{p}{,} \PY{n}{ewnp10} \PY{o}{=} \PY{l+m+mi}{0}\PY{p}{,}\PY{l+m+mi}{0} \PY{c+c1}{\PYZsh{} W | \PYZca{}\PYZob{}n+1\PYZcb{} \PYZus{}\PYZob{}0\PYZcb{}}
        \PY{n}{hwnm11}\PY{p}{,} \PY{n}{ewnm11} \PY{o}{=} \PY{l+m+mi}{0}\PY{p}{,}\PY{l+m+mi}{0} \PY{c+c1}{\PYZsh{} W | \PYZca{}\PYZob{}n\PYZhy{}1\PYZcb{} \PYZus{}\PYZob{}1\PYZcb{}}
        \PY{n}{hwnp11}\PY{p}{,} \PY{n}{ewnp11} \PY{o}{=} \PY{l+m+mi}{0}\PY{p}{,}\PY{l+m+mi}{0} \PY{c+c1}{\PYZsh{} W | \PYZca{}\PYZob{}n+1\PYZcb{} \PYZus{}\PYZob{}1\PYZcb{}}
        \PY{n}{hwnm10}\PY{p}{,} \PY{n}{ewnm10} \PY{o}{=} \PY{l+m+mi}{0}\PY{p}{,}\PY{l+m+mi}{0} \PY{c+c1}{\PYZsh{} W | \PYZca{}\PYZob{}n\PYZhy{}1\PYZcb{} \PYZus{}\PYZob{}0\PYZcb{}}
        \PY{n}{hwn0}  \PY{p}{,} \PY{n}{ewn0}   \PY{o}{=} \PY{l+m+mi}{0}\PY{p}{,}\PY{l+m+mi}{0} \PY{c+c1}{\PYZsh{} W | \PYZca{}\PYZob{}n  \PYZcb{} \PYZus{}\PYZob{}0\PYZcb{}}
        \PY{n}{hwn1}  \PY{p}{,} \PY{n}{ewn1}   \PY{o}{=} \PY{l+m+mi}{0}\PY{p}{,}\PY{l+m+mi}{0} \PY{c+c1}{\PYZsh{} W | \PYZca{}\PYZob{}n  \PYZcb{} \PYZus{}\PYZob{}1\PYZcb{}}
        
        \PY{n}{hwnp1im1}\PY{p}{,} \PY{n}{ewnp1im1} \PY{o}{=} \PY{l+m+mi}{0}\PY{p}{,}\PY{l+m+mi}{0} \PY{c+c1}{\PYZsh{} W | \PYZca{}\PYZob{}n+1\PYZcb{} \PYZus{}\PYZob{}i\PYZhy{}1\PYZcb{}}
        \PY{n}{hwnm1i}  \PY{p}{,} \PY{n}{ewnm1i}   \PY{o}{=} \PY{l+m+mi}{0}\PY{p}{,}\PY{l+m+mi}{0} \PY{c+c1}{\PYZsh{} W | \PYZca{}\PYZob{}n\PYZhy{}1\PYZcb{} \PYZus{}\PYZob{}i  \PYZcb{}}
        \PY{n}{hwnp1i}  \PY{p}{,} \PY{n}{ewnp1i}   \PY{o}{=} \PY{l+m+mi}{0}\PY{p}{,}\PY{l+m+mi}{0} \PY{c+c1}{\PYZsh{} W | \PYZca{}\PYZob{}n+1\PYZcb{} \PYZus{}\PYZob{}i  \PYZcb{}}
        \PY{n}{hwnm1im1}\PY{p}{,} \PY{n}{ewnm1im1} \PY{o}{=} \PY{l+m+mi}{0}\PY{p}{,}\PY{l+m+mi}{0} \PY{c+c1}{\PYZsh{} W | \PYZca{}\PYZob{}n\PYZhy{}1\PYZcb{} \PYZus{}\PYZob{}i\PYZhy{}1\PYZcb{}}
        \PY{n}{hwnim1}  \PY{p}{,} \PY{n}{ewnim1}   \PY{o}{=} \PY{l+m+mi}{0}\PY{p}{,}\PY{l+m+mi}{0} \PY{c+c1}{\PYZsh{} W | \PYZca{}\PYZob{}n  \PYZcb{} \PYZus{}\PYZob{}i\PYZhy{}1\PYZcb{}}
        \PY{n}{hwni}    \PY{p}{,} \PY{n}{ewni}     \PY{o}{=} \PY{l+m+mi}{0}\PY{p}{,}\PY{l+m+mi}{0} \PY{c+c1}{\PYZsh{} W | \PYZca{}\PYZob{}n  \PYZcb{} \PYZus{}\PYZob{}i  \PYZcb{}}
\end{Verbatim}

    For the actual loop, we begin with modifying the loop of the field with
an additive source from the previous task. The source is a Gaussian
field at the origin, defined here in line 24. The electric and magnetic
fields propagate in lines 5 and 24.

In applying the boundary conditions at the left, we begin with setting
the variable \(H \Big |^{n+1} _{1}\) (\(E \Big |^{n+1} _{1}\)) using the
field that was obtained in line 5(24). then we use the equation that we
derived for \(H \Big |^{n+1} _{0}\)(\$ E \Big \textbar{}\^{}\{n+1\}
\_\{0\}\$) in line 9(29). This then becomes the initial value of the
field at \(x=0\) for the next iteration. We then cycle through the
values of \(H\)(\(E\)) which will be used for the next run of the loop.

Similarly to the right, we begin with
\(H \Big |^{n+1} _{i-1}\)(\(E \Big |^{n+1} _{i-1}\)), and assign to this
the value of the field at at \(nx-2\) (note that the field goes only
from \(0\) to \(nx-1\)). In order to obtain the field at \(nx-1\), we
use the previously derived equation and assign this to the array at
\(x=nx-1\). Then we also cycle through the values of \(H\)(\(E\)) for
the next iteration.

The code starting from line 41 is used to plot the fields.

    \begin{Verbatim}[commandchars=\\\{\}]
{\color{incolor}In [{\color{incolor}7}]:} \PY{k}{for} \PY{n}{dt} \PY{o+ow}{in} \PY{n+nb}{range}\PY{p}{(}\PY{l+m+mi}{0}\PY{p}{,}\PY{n}{nt}\PY{p}{)}\PY{p}{:}
            \PY{c+c1}{\PYZsh{}\PYZsh{}\PYZsh{}\PYZsh{}\PYZsh{}\PYZsh{}\PYZsh{}\PYZsh{}\PYZsh{}\PYZsh{}\PYZsh{}\PYZsh{}\PYZsh{}\PYZsh{}\PYZsh{}\PYZsh{}\PYZsh{}\PYZsh{}\PYZsh{}\PYZsh{}\PYZsh{}\PYZsh{}}
            \PY{c+c1}{\PYZsh{}Magnetic field}
            \PY{c+c1}{\PYZsh{}\PYZsh{}\PYZsh{}\PYZsh{}\PYZsh{}\PYZsh{}\PYZsh{}\PYZsh{}\PYZsh{}\PYZsh{}\PYZsh{}\PYZsh{}\PYZsh{}\PYZsh{}\PYZsh{}\PYZsh{}\PYZsh{}\PYZsh{}\PYZsh{}\PYZsh{}\PYZsh{}\PYZsh{}}
            \PY{n}{hy}\PY{p}{[}\PY{n}{x}\PY{p}{]} \PY{o}{=} \PY{n}{hy}\PY{p}{[}\PY{n}{x}\PY{p}{]} \PY{o}{+} \PY{p}{(}\PY{n}{ez}\PY{p}{[}\PY{n}{x}\PY{o}{+}\PY{l+m+mi}{1}\PY{p}{]} \PY{o}{\PYZhy{}} \PY{n}{ez}\PY{p}{[}\PY{n}{x}\PY{p}{]}\PY{p}{)}\PY{o}{/}\PY{n}{imp0}
        
            \PY{c+c1}{\PYZsh{}abc at left}
            \PY{n}{hwnp11} \PY{o}{=} \PY{n}{hy}\PY{p}{[}\PY{l+m+mi}{1}\PY{p}{]}
            \PY{n}{hwnp10} \PY{o}{=} \PY{o}{\PYZhy{}}\PY{n}{hwnm11} \PY{o}{+} \PY{n}{a}\PY{o}{*}\PY{p}{(}\PY{n}{hwnp11} \PY{o}{+} \PY{n}{hwnm10}\PY{p}{)} \PY{o}{+} \PY{n}{b}\PY{o}{*}\PY{p}{(}\PY{n}{hwn0} \PY{o}{+} \PY{n}{hwn1}\PY{p}{)}
            \PY{n}{hy}\PY{p}{[}\PY{l+m+mi}{0}\PY{p}{]} \PY{o}{=} \PY{n}{hwnp10}
            \PY{n}{hwnm11}\PY{p}{,} \PY{n}{hwnm10} \PY{o}{=} \PY{n}{hwn1}\PY{p}{,} \PY{n}{hwn0}
            \PY{n}{hwn1}\PY{p}{,} \PY{n}{hwn0}  \PY{o}{=} \PY{n}{hwnp11}\PY{p}{,} \PY{n}{hwnp10}
            
            \PY{c+c1}{\PYZsh{}abc at right}
            \PY{n}{hwnp1im1} \PY{o}{=} \PY{n}{hy}\PY{p}{[}\PY{o}{\PYZhy{}}\PY{l+m+mi}{2}\PY{p}{]}
            \PY{n}{hwnp1i} \PY{o}{=} \PY{o}{\PYZhy{}} \PY{n}{hwnm1im1} \PY{o}{+} \PY{n}{a}\PY{o}{*}\PY{p}{(}\PY{n}{hwnp1im1} \PY{o}{+} \PY{n}{hwnm1i}\PY{p}{)} \PY{o}{+} \PY{n}{b}\PY{o}{*}\PY{p}{(}\PY{n}{hwnp1i} \PY{o}{+} \PY{n}{hwnim1}\PY{p}{)}
            \PY{n}{hy}\PY{p}{[}\PY{o}{\PYZhy{}}\PY{l+m+mi}{1}\PY{p}{]} \PY{o}{=} \PY{n}{hwnp1i}
            \PY{n}{hwnm1i}\PY{p}{,} \PY{n}{hwnm1im1} \PY{o}{=} \PY{n}{hwni}\PY{p}{,} \PY{n}{hwnim1}
            \PY{n}{hwni}\PY{p}{,} \PY{n}{hwnim1}  \PY{o}{=} \PY{n}{hwnp1i}\PY{p}{,} \PY{n}{hwnp1im1}
            
            \PY{c+c1}{\PYZsh{}\PYZsh{}\PYZsh{}\PYZsh{}\PYZsh{}\PYZsh{}\PYZsh{}\PYZsh{}\PYZsh{}\PYZsh{}\PYZsh{}\PYZsh{}\PYZsh{}\PYZsh{}\PYZsh{}\PYZsh{}\PYZsh{}\PYZsh{}\PYZsh{}\PYZsh{}\PYZsh{}\PYZsh{}}
            \PY{c+c1}{\PYZsh{}Electric field}
            \PY{c+c1}{\PYZsh{}\PYZsh{}\PYZsh{}\PYZsh{}\PYZsh{}\PYZsh{}\PYZsh{}\PYZsh{}\PYZsh{}\PYZsh{}\PYZsh{}\PYZsh{}\PYZsh{}\PYZsh{}\PYZsh{}\PYZsh{}\PYZsh{}\PYZsh{}\PYZsh{}\PYZsh{}\PYZsh{}\PYZsh{}}
            \PY{n}{ez}\PY{p}{[}\PY{n}{x}\PY{o}{+}\PY{l+m+mi}{1}\PY{p}{]} \PY{o}{=} \PY{n}{ez}\PY{p}{[}\PY{n}{x}\PY{o}{+}\PY{l+m+mi}{1}\PY{p}{]} \PY{o}{+} \PY{p}{(}\PY{n}{hy}\PY{p}{[}\PY{n}{x}\PY{o}{+}\PY{l+m+mi}{1}\PY{p}{]}\PY{o}{\PYZhy{}}\PY{n}{hy}\PY{p}{[}\PY{n}{x}\PY{p}{]}\PY{p}{)}\PY{o}{*}\PY{n}{imp0}\PY{o}{/}\PY{n}{epsilon}
            \PY{n}{ez}\PY{p}{[}\PY{n}{srcori}\PY{p}{]} \PY{o}{+}\PY{o}{=} \PY{n}{m}\PY{o}{.}\PY{n}{exp}\PY{p}{(}\PY{o}{\PYZhy{}}\PY{p}{(}\PY{p}{(}\PY{n}{dt}\PY{o}{\PYZhy{}}\PY{n}{srcdel}\PY{p}{)}\PY{o}{*}\PY{p}{(}\PY{n}{dt}\PY{o}{\PYZhy{}}\PY{n}{srcdel}\PY{p}{)}\PY{p}{)}\PY{o}{/}\PY{p}{(}\PY{n}{srcwid}\PY{o}{*}\PY{n}{srcwid}\PY{p}{)}\PY{p}{)}
        
            \PY{c+c1}{\PYZsh{}abc at left}
            \PY{n}{ewnp11} \PY{o}{=} \PY{n}{ez}\PY{p}{[}\PY{l+m+mi}{1}\PY{p}{]}
            \PY{n}{ewnp10} \PY{o}{=} \PY{o}{\PYZhy{}}\PY{n}{ewnm11} \PY{o}{+} \PY{n}{a}\PY{o}{*}\PY{p}{(}\PY{n}{ewnp11} \PY{o}{+} \PY{n}{ewnm10}\PY{p}{)} \PY{o}{+} \PY{n}{b}\PY{o}{*}\PY{p}{(}\PY{n}{ewn0} \PY{o}{+} \PY{n}{ewn1}\PY{p}{)}
            \PY{n}{ez}\PY{p}{[}\PY{l+m+mi}{0}\PY{p}{]} \PY{o}{=} \PY{n}{ewnp10}
            \PY{n}{ewnm11}\PY{p}{,} \PY{n}{ewnm10} \PY{o}{=} \PY{n}{ewn1}\PY{p}{,} \PY{n}{ewn0}
            \PY{n}{ewn1}\PY{p}{,} \PY{n}{ewn0}  \PY{o}{=} \PY{n}{ewnp11}\PY{p}{,} \PY{n}{ewnp10}
            
            \PY{c+c1}{\PYZsh{}abc at right}
            \PY{n}{ewnp1im1} \PY{o}{=} \PY{n}{ez}\PY{p}{[}\PY{o}{\PYZhy{}}\PY{l+m+mi}{2}\PY{p}{]}
            \PY{n}{ewnp1i} \PY{o}{=} \PY{o}{\PYZhy{}} \PY{n}{ewnm1im1} \PY{o}{+} \PY{n}{a}\PY{o}{*}\PY{p}{(}\PY{n}{ewnp1im1} \PY{o}{+} \PY{n}{ewnm1i}\PY{p}{)} \PY{o}{+} \PY{n}{b}\PY{o}{*}\PY{p}{(}\PY{n}{ewnp1i} \PY{o}{+} \PY{n}{ewnim1}\PY{p}{)}
            \PY{n}{ez}\PY{p}{[}\PY{o}{\PYZhy{}}\PY{l+m+mi}{1}\PY{p}{]} \PY{o}{=} \PY{n}{ewnp1i}
            \PY{n}{ewnm1i}\PY{p}{,} \PY{n}{ewnm1im1} \PY{o}{=} \PY{n}{ewni}\PY{p}{,} \PY{n}{ewnim1}
            \PY{n}{ewni}\PY{p}{,} \PY{n}{ewnim1}  \PY{o}{=} \PY{n}{ewnp1i}\PY{p}{,} \PY{n}{ewnp1im1}
        
            \PY{n}{plt}\PY{o}{.}\PY{n}{hold}\PY{p}{(}\PY{k+kc}{True}\PY{p}{)}
            \PY{k}{if} \PY{p}{(}\PY{n}{dt} \PY{o}{\PYZpc{}} \PY{l+m+mi}{60} \PY{o}{==} \PY{l+m+mi}{0} \PY{o+ow}{or} \PY{n}{dt} \PY{o}{==} \PY{n}{nt}\PY{o}{\PYZhy{}}\PY{l+m+mi}{2}\PY{p}{)}\PY{p}{:}
                \PY{n}{fignum} \PY{o}{=} \PY{n}{fignum} \PY{o}{+} \PY{l+m+mi}{1}
                \PY{n}{plt}\PY{o}{.}\PY{n}{figure}\PY{p}{(}\PY{n}{fignum}\PY{p}{)}
                \PY{n}{plt}\PY{o}{.}\PY{n}{xlabel}\PY{p}{(}\PY{l+s+s2}{\PYZdq{}}\PY{l+s+s2}{Position}\PY{l+s+s2}{\PYZdq{}}\PY{p}{)}
                \PY{n}{plt}\PY{o}{.}\PY{n}{ylabel}\PY{p}{(}\PY{l+s+s2}{\PYZdq{}}\PY{l+s+s2}{Amplitude}\PY{l+s+s2}{\PYZdq{}}\PY{p}{)}
                \PY{n}{plt}\PY{o}{.}\PY{n}{title}\PY{p}{(}\PY{l+s+s2}{\PYZdq{}}\PY{l+s+s2}{Field at t = }\PY{l+s+s2}{\PYZdq{}}\PY{o}{+} \PY{n+nb}{str}\PY{p}{(}\PY{n}{dt}\PY{p}{)}\PY{p}{)}
                \PY{n}{plt}\PY{o}{.}\PY{n}{plot}\PY{p}{(}\PY{n}{ez}\PY{p}{,} \PY{n}{label}\PY{o}{=}\PY{l+s+s2}{\PYZdq{}}\PY{l+s+s2}{E\PYZhy{}field}\PY{l+s+s2}{\PYZdq{}}\PY{p}{)}
                \PY{n}{plt}\PY{o}{.}\PY{n}{plot}\PY{p}{(}\PY{n}{hy}\PY{o}{*}\PY{n}{imp0}\PY{p}{,} \PY{n}{label}\PY{o}{=}\PY{l+s+s2}{\PYZdq{}}\PY{l+s+s2}{H\PYZhy{}field}\PY{l+s+s2}{\PYZdq{}}\PY{p}{)}
                \PY{n}{plt}\PY{o}{.}\PY{n}{legend}\PY{p}{(}\PY{p}{)}
\end{Verbatim}

    \begin{center}
    \adjustimage{max size={0.9\linewidth}{0.9\paperheight}}{task1_files/task1_17_0.png}
    \end{center}
    { \hspace*{\fill} \\}
    
    \begin{center}
    \adjustimage{max size={0.9\linewidth}{0.9\paperheight}}{task1_files/task1_17_1.png}
    \end{center}
    { \hspace*{\fill} \\}
    
    \begin{center}
    \adjustimage{max size={0.9\linewidth}{0.9\paperheight}}{task1_files/task1_17_2.png}
    \end{center}
    { \hspace*{\fill} \\}
    
    \begin{center}
    \adjustimage{max size={0.9\linewidth}{0.9\paperheight}}{task1_files/task1_17_3.png}
    \end{center}
    { \hspace*{\fill} \\}
    
    \begin{center}
    \adjustimage{max size={0.9\linewidth}{0.9\paperheight}}{task1_files/task1_17_4.png}
    \end{center}
    { \hspace*{\fill} \\}
    
    \begin{center}
    \adjustimage{max size={0.9\linewidth}{0.9\paperheight}}{task1_files/task1_17_5.png}
    \end{center}
    { \hspace*{\fill} \\}
    
    \begin{center}
    \adjustimage{max size={0.9\linewidth}{0.9\paperheight}}{task1_files/task1_17_6.png}
    \end{center}
    { \hspace*{\fill} \\}
    
    \begin{center}
    \adjustimage{max size={0.9\linewidth}{0.9\paperheight}}{task1_files/task1_17_7.png}
    \end{center}
    { \hspace*{\fill} \\}
    
    \begin{center}
    \adjustimage{max size={0.9\linewidth}{0.9\paperheight}}{task1_files/task1_17_8.png}
    \end{center}
    { \hspace*{\fill} \\}
    
    \begin{center}
    \adjustimage{max size={0.9\linewidth}{0.9\paperheight}}{task1_files/task1_17_9.png}
    \end{center}
    { \hspace*{\fill} \\}
    
    \begin{center}
    \adjustimage{max size={0.9\linewidth}{0.9\paperheight}}{task1_files/task1_17_10.png}
    \end{center}
    { \hspace*{\fill} \\}
    
    \begin{center}
    \adjustimage{max size={0.9\linewidth}{0.9\paperheight}}{task1_files/task1_17_11.png}
    \end{center}
    { \hspace*{\fill} \\}
    
    \begin{center}
    \adjustimage{max size={0.9\linewidth}{0.9\paperheight}}{task1_files/task1_17_12.png}
    \end{center}
    { \hspace*{\fill} \\}
    
    \begin{center}
    \adjustimage{max size={0.9\linewidth}{0.9\paperheight}}{task1_files/task1_17_13.png}
    \end{center}
    { \hspace*{\fill} \\}
    
    \section{References}\label{references}

Taflove, A. \& Hagness, S. C. \emph{Computational electrodynamics: the
finite-difference time-domain method}. (Artech House, 2005).

Engquist, B. \& Majda, A. \emph{Absorbing boundary conditions for
numerical simulation of waves}. Proc Natl Acad Sci U S A 74, 1765--1766
(1977).


    % Add a bibliography block to the postdoc
    
    
    
    \end{document}
