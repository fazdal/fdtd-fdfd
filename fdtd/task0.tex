
% Default to the notebook output style

    


% Inherit from the specified cell style.




    
\documentclass{article}

    
    
    \usepackage{graphicx} % Used to insert images
    \usepackage{adjustbox} % Used to constrain images to a maximum size 
    \usepackage{color} % Allow colors to be defined
    \usepackage{enumerate} % Needed for markdown enumerations to work
    \usepackage{geometry} % Used to adjust the document margins
    \usepackage{amsmath} % Equations
    \usepackage{amssymb} % Equations
    \usepackage{eurosym} % defines \euro
    \usepackage[mathletters]{ucs} % Extended unicode (utf-8) support
    \usepackage[utf8x]{inputenc} % Allow utf-8 characters in the tex document
    \usepackage{fancyvrb} % verbatim replacement that allows latex
    \usepackage{grffile} % extends the file name processing of package graphics 
                         % to support a larger range 
    % The hyperref package gives us a pdf with properly built
    % internal navigation ('pdf bookmarks' for the table of contents,
    % internal cross-reference links, web links for URLs, etc.)
    \usepackage{hyperref}
    \usepackage{longtable} % longtable support required by pandoc >1.10
    \usepackage{booktabs}  % table support for pandoc > 1.12.2
    \usepackage{ulem} % ulem is needed to support strikethroughs (\sout)
    

    
    
    \definecolor{orange}{cmyk}{0,0.4,0.8,0.2}
    \definecolor{darkorange}{rgb}{.71,0.21,0.01}
    \definecolor{darkgreen}{rgb}{.12,.54,.11}
    \definecolor{myteal}{rgb}{.26, .44, .56}
    \definecolor{gray}{gray}{0.45}
    \definecolor{lightgray}{gray}{.95}
    \definecolor{mediumgray}{gray}{.8}
    \definecolor{inputbackground}{rgb}{.95, .95, .85}
    \definecolor{outputbackground}{rgb}{.95, .95, .95}
    \definecolor{traceback}{rgb}{1, .95, .95}
    % ansi colors
    \definecolor{red}{rgb}{.6,0,0}
    \definecolor{green}{rgb}{0,.65,0}
    \definecolor{brown}{rgb}{0.6,0.6,0}
    \definecolor{blue}{rgb}{0,.145,.698}
    \definecolor{purple}{rgb}{.698,.145,.698}
    \definecolor{cyan}{rgb}{0,.698,.698}
    \definecolor{lightgray}{gray}{0.5}
    
    % bright ansi colors
    \definecolor{darkgray}{gray}{0.25}
    \definecolor{lightred}{rgb}{1.0,0.39,0.28}
    \definecolor{lightgreen}{rgb}{0.48,0.99,0.0}
    \definecolor{lightblue}{rgb}{0.53,0.81,0.92}
    \definecolor{lightpurple}{rgb}{0.87,0.63,0.87}
    \definecolor{lightcyan}{rgb}{0.5,1.0,0.83}
    
    % commands and environments needed by pandoc snippets
    % extracted from the output of `pandoc -s`
    \providecommand{\tightlist}{%
      \setlength{\itemsep}{0pt}\setlength{\parskip}{0pt}}
    \DefineVerbatimEnvironment{Highlighting}{Verbatim}{commandchars=\\\{\}}
    % Add ',fontsize=\small' for more characters per line
    \newenvironment{Shaded}{}{}
    \newcommand{\KeywordTok}[1]{\textcolor[rgb]{0.00,0.44,0.13}{\textbf{{#1}}}}
    \newcommand{\DataTypeTok}[1]{\textcolor[rgb]{0.56,0.13,0.00}{{#1}}}
    \newcommand{\DecValTok}[1]{\textcolor[rgb]{0.25,0.63,0.44}{{#1}}}
    \newcommand{\BaseNTok}[1]{\textcolor[rgb]{0.25,0.63,0.44}{{#1}}}
    \newcommand{\FloatTok}[1]{\textcolor[rgb]{0.25,0.63,0.44}{{#1}}}
    \newcommand{\CharTok}[1]{\textcolor[rgb]{0.25,0.44,0.63}{{#1}}}
    \newcommand{\StringTok}[1]{\textcolor[rgb]{0.25,0.44,0.63}{{#1}}}
    \newcommand{\CommentTok}[1]{\textcolor[rgb]{0.38,0.63,0.69}{\textit{{#1}}}}
    \newcommand{\OtherTok}[1]{\textcolor[rgb]{0.00,0.44,0.13}{{#1}}}
    \newcommand{\AlertTok}[1]{\textcolor[rgb]{1.00,0.00,0.00}{\textbf{{#1}}}}
    \newcommand{\FunctionTok}[1]{\textcolor[rgb]{0.02,0.16,0.49}{{#1}}}
    \newcommand{\RegionMarkerTok}[1]{{#1}}
    \newcommand{\ErrorTok}[1]{\textcolor[rgb]{1.00,0.00,0.00}{\textbf{{#1}}}}
    \newcommand{\NormalTok}[1]{{#1}}
    
    % Additional commands for more recent versions of Pandoc
    \newcommand{\ConstantTok}[1]{\textcolor[rgb]{0.53,0.00,0.00}{{#1}}}
    \newcommand{\SpecialCharTok}[1]{\textcolor[rgb]{0.25,0.44,0.63}{{#1}}}
    \newcommand{\VerbatimStringTok}[1]{\textcolor[rgb]{0.25,0.44,0.63}{{#1}}}
    \newcommand{\SpecialStringTok}[1]{\textcolor[rgb]{0.73,0.40,0.53}{{#1}}}
    \newcommand{\ImportTok}[1]{{#1}}
    \newcommand{\DocumentationTok}[1]{\textcolor[rgb]{0.73,0.13,0.13}{\textit{{#1}}}}
    \newcommand{\AnnotationTok}[1]{\textcolor[rgb]{0.38,0.63,0.69}{\textbf{\textit{{#1}}}}}
    \newcommand{\CommentVarTok}[1]{\textcolor[rgb]{0.38,0.63,0.69}{\textbf{\textit{{#1}}}}}
    \newcommand{\VariableTok}[1]{\textcolor[rgb]{0.10,0.09,0.49}{{#1}}}
    \newcommand{\ControlFlowTok}[1]{\textcolor[rgb]{0.00,0.44,0.13}{\textbf{{#1}}}}
    \newcommand{\OperatorTok}[1]{\textcolor[rgb]{0.40,0.40,0.40}{{#1}}}
    \newcommand{\BuiltInTok}[1]{{#1}}
    \newcommand{\ExtensionTok}[1]{{#1}}
    \newcommand{\PreprocessorTok}[1]{\textcolor[rgb]{0.74,0.48,0.00}{{#1}}}
    \newcommand{\AttributeTok}[1]{\textcolor[rgb]{0.49,0.56,0.16}{{#1}}}
    \newcommand{\InformationTok}[1]{\textcolor[rgb]{0.38,0.63,0.69}{\textbf{\textit{{#1}}}}}
    \newcommand{\WarningTok}[1]{\textcolor[rgb]{0.38,0.63,0.69}{\textbf{\textit{{#1}}}}}
    
    
    % Define a nice break command that doesn't care if a line doesn't already
    % exist.
    \def\br{\hspace*{\fill} \\* }
    % Math Jax compatability definitions
    \def\gt{>}
    \def\lt{<}
    % Document parameters
    \title{Task 0: Vanishing EM Fields}
    \author{Ang, Angeleene S.}
    
    

    % Pygments definitions
    
\makeatletter
\def\PY@reset{\let\PY@it=\relax \let\PY@bf=\relax%
    \let\PY@ul=\relax \let\PY@tc=\relax%
    \let\PY@bc=\relax \let\PY@ff=\relax}
\def\PY@tok#1{\csname PY@tok@#1\endcsname}
\def\PY@toks#1+{\ifx\relax#1\empty\else%
    \PY@tok{#1}\expandafter\PY@toks\fi}
\def\PY@do#1{\PY@bc{\PY@tc{\PY@ul{%
    \PY@it{\PY@bf{\PY@ff{#1}}}}}}}
\def\PY#1#2{\PY@reset\PY@toks#1+\relax+\PY@do{#2}}

\expandafter\def\csname PY@tok@no\endcsname{\def\PY@tc##1{\textcolor[rgb]{0.53,0.00,0.00}{##1}}}
\expandafter\def\csname PY@tok@gh\endcsname{\let\PY@bf=\textbf\def\PY@tc##1{\textcolor[rgb]{0.00,0.00,0.50}{##1}}}
\expandafter\def\csname PY@tok@cp\endcsname{\def\PY@tc##1{\textcolor[rgb]{0.74,0.48,0.00}{##1}}}
\expandafter\def\csname PY@tok@gp\endcsname{\let\PY@bf=\textbf\def\PY@tc##1{\textcolor[rgb]{0.00,0.00,0.50}{##1}}}
\expandafter\def\csname PY@tok@sd\endcsname{\let\PY@it=\textit\def\PY@tc##1{\textcolor[rgb]{0.73,0.13,0.13}{##1}}}
\expandafter\def\csname PY@tok@cpf\endcsname{\let\PY@it=\textit\def\PY@tc##1{\textcolor[rgb]{0.25,0.50,0.50}{##1}}}
\expandafter\def\csname PY@tok@nd\endcsname{\def\PY@tc##1{\textcolor[rgb]{0.67,0.13,1.00}{##1}}}
\expandafter\def\csname PY@tok@mo\endcsname{\def\PY@tc##1{\textcolor[rgb]{0.40,0.40,0.40}{##1}}}
\expandafter\def\csname PY@tok@gs\endcsname{\let\PY@bf=\textbf}
\expandafter\def\csname PY@tok@kc\endcsname{\let\PY@bf=\textbf\def\PY@tc##1{\textcolor[rgb]{0.00,0.50,0.00}{##1}}}
\expandafter\def\csname PY@tok@s1\endcsname{\def\PY@tc##1{\textcolor[rgb]{0.73,0.13,0.13}{##1}}}
\expandafter\def\csname PY@tok@kp\endcsname{\def\PY@tc##1{\textcolor[rgb]{0.00,0.50,0.00}{##1}}}
\expandafter\def\csname PY@tok@sr\endcsname{\def\PY@tc##1{\textcolor[rgb]{0.73,0.40,0.53}{##1}}}
\expandafter\def\csname PY@tok@vc\endcsname{\def\PY@tc##1{\textcolor[rgb]{0.10,0.09,0.49}{##1}}}
\expandafter\def\csname PY@tok@kr\endcsname{\let\PY@bf=\textbf\def\PY@tc##1{\textcolor[rgb]{0.00,0.50,0.00}{##1}}}
\expandafter\def\csname PY@tok@k\endcsname{\let\PY@bf=\textbf\def\PY@tc##1{\textcolor[rgb]{0.00,0.50,0.00}{##1}}}
\expandafter\def\csname PY@tok@il\endcsname{\def\PY@tc##1{\textcolor[rgb]{0.40,0.40,0.40}{##1}}}
\expandafter\def\csname PY@tok@nv\endcsname{\def\PY@tc##1{\textcolor[rgb]{0.10,0.09,0.49}{##1}}}
\expandafter\def\csname PY@tok@w\endcsname{\def\PY@tc##1{\textcolor[rgb]{0.73,0.73,0.73}{##1}}}
\expandafter\def\csname PY@tok@mh\endcsname{\def\PY@tc##1{\textcolor[rgb]{0.40,0.40,0.40}{##1}}}
\expandafter\def\csname PY@tok@gt\endcsname{\def\PY@tc##1{\textcolor[rgb]{0.00,0.27,0.87}{##1}}}
\expandafter\def\csname PY@tok@kn\endcsname{\let\PY@bf=\textbf\def\PY@tc##1{\textcolor[rgb]{0.00,0.50,0.00}{##1}}}
\expandafter\def\csname PY@tok@mf\endcsname{\def\PY@tc##1{\textcolor[rgb]{0.40,0.40,0.40}{##1}}}
\expandafter\def\csname PY@tok@o\endcsname{\def\PY@tc##1{\textcolor[rgb]{0.40,0.40,0.40}{##1}}}
\expandafter\def\csname PY@tok@ow\endcsname{\let\PY@bf=\textbf\def\PY@tc##1{\textcolor[rgb]{0.67,0.13,1.00}{##1}}}
\expandafter\def\csname PY@tok@go\endcsname{\def\PY@tc##1{\textcolor[rgb]{0.53,0.53,0.53}{##1}}}
\expandafter\def\csname PY@tok@ch\endcsname{\let\PY@it=\textit\def\PY@tc##1{\textcolor[rgb]{0.25,0.50,0.50}{##1}}}
\expandafter\def\csname PY@tok@mi\endcsname{\def\PY@tc##1{\textcolor[rgb]{0.40,0.40,0.40}{##1}}}
\expandafter\def\csname PY@tok@s\endcsname{\def\PY@tc##1{\textcolor[rgb]{0.73,0.13,0.13}{##1}}}
\expandafter\def\csname PY@tok@gi\endcsname{\def\PY@tc##1{\textcolor[rgb]{0.00,0.63,0.00}{##1}}}
\expandafter\def\csname PY@tok@m\endcsname{\def\PY@tc##1{\textcolor[rgb]{0.40,0.40,0.40}{##1}}}
\expandafter\def\csname PY@tok@nn\endcsname{\let\PY@bf=\textbf\def\PY@tc##1{\textcolor[rgb]{0.00,0.00,1.00}{##1}}}
\expandafter\def\csname PY@tok@mb\endcsname{\def\PY@tc##1{\textcolor[rgb]{0.40,0.40,0.40}{##1}}}
\expandafter\def\csname PY@tok@nb\endcsname{\def\PY@tc##1{\textcolor[rgb]{0.00,0.50,0.00}{##1}}}
\expandafter\def\csname PY@tok@ni\endcsname{\let\PY@bf=\textbf\def\PY@tc##1{\textcolor[rgb]{0.60,0.60,0.60}{##1}}}
\expandafter\def\csname PY@tok@c1\endcsname{\let\PY@it=\textit\def\PY@tc##1{\textcolor[rgb]{0.25,0.50,0.50}{##1}}}
\expandafter\def\csname PY@tok@ss\endcsname{\def\PY@tc##1{\textcolor[rgb]{0.10,0.09,0.49}{##1}}}
\expandafter\def\csname PY@tok@sc\endcsname{\def\PY@tc##1{\textcolor[rgb]{0.73,0.13,0.13}{##1}}}
\expandafter\def\csname PY@tok@sh\endcsname{\def\PY@tc##1{\textcolor[rgb]{0.73,0.13,0.13}{##1}}}
\expandafter\def\csname PY@tok@nl\endcsname{\def\PY@tc##1{\textcolor[rgb]{0.63,0.63,0.00}{##1}}}
\expandafter\def\csname PY@tok@gd\endcsname{\def\PY@tc##1{\textcolor[rgb]{0.63,0.00,0.00}{##1}}}
\expandafter\def\csname PY@tok@gu\endcsname{\let\PY@bf=\textbf\def\PY@tc##1{\textcolor[rgb]{0.50,0.00,0.50}{##1}}}
\expandafter\def\csname PY@tok@sx\endcsname{\def\PY@tc##1{\textcolor[rgb]{0.00,0.50,0.00}{##1}}}
\expandafter\def\csname PY@tok@se\endcsname{\let\PY@bf=\textbf\def\PY@tc##1{\textcolor[rgb]{0.73,0.40,0.13}{##1}}}
\expandafter\def\csname PY@tok@nt\endcsname{\let\PY@bf=\textbf\def\PY@tc##1{\textcolor[rgb]{0.00,0.50,0.00}{##1}}}
\expandafter\def\csname PY@tok@c\endcsname{\let\PY@it=\textit\def\PY@tc##1{\textcolor[rgb]{0.25,0.50,0.50}{##1}}}
\expandafter\def\csname PY@tok@vg\endcsname{\def\PY@tc##1{\textcolor[rgb]{0.10,0.09,0.49}{##1}}}
\expandafter\def\csname PY@tok@gr\endcsname{\def\PY@tc##1{\textcolor[rgb]{1.00,0.00,0.00}{##1}}}
\expandafter\def\csname PY@tok@cm\endcsname{\let\PY@it=\textit\def\PY@tc##1{\textcolor[rgb]{0.25,0.50,0.50}{##1}}}
\expandafter\def\csname PY@tok@nf\endcsname{\def\PY@tc##1{\textcolor[rgb]{0.00,0.00,1.00}{##1}}}
\expandafter\def\csname PY@tok@nc\endcsname{\let\PY@bf=\textbf\def\PY@tc##1{\textcolor[rgb]{0.00,0.00,1.00}{##1}}}
\expandafter\def\csname PY@tok@si\endcsname{\let\PY@bf=\textbf\def\PY@tc##1{\textcolor[rgb]{0.73,0.40,0.53}{##1}}}
\expandafter\def\csname PY@tok@ge\endcsname{\let\PY@it=\textit}
\expandafter\def\csname PY@tok@sb\endcsname{\def\PY@tc##1{\textcolor[rgb]{0.73,0.13,0.13}{##1}}}
\expandafter\def\csname PY@tok@s2\endcsname{\def\PY@tc##1{\textcolor[rgb]{0.73,0.13,0.13}{##1}}}
\expandafter\def\csname PY@tok@vi\endcsname{\def\PY@tc##1{\textcolor[rgb]{0.10,0.09,0.49}{##1}}}
\expandafter\def\csname PY@tok@kt\endcsname{\def\PY@tc##1{\textcolor[rgb]{0.69,0.00,0.25}{##1}}}
\expandafter\def\csname PY@tok@kd\endcsname{\let\PY@bf=\textbf\def\PY@tc##1{\textcolor[rgb]{0.00,0.50,0.00}{##1}}}
\expandafter\def\csname PY@tok@cs\endcsname{\let\PY@it=\textit\def\PY@tc##1{\textcolor[rgb]{0.25,0.50,0.50}{##1}}}
\expandafter\def\csname PY@tok@ne\endcsname{\let\PY@bf=\textbf\def\PY@tc##1{\textcolor[rgb]{0.82,0.25,0.23}{##1}}}
\expandafter\def\csname PY@tok@bp\endcsname{\def\PY@tc##1{\textcolor[rgb]{0.00,0.50,0.00}{##1}}}
\expandafter\def\csname PY@tok@err\endcsname{\def\PY@bc##1{\setlength{\fboxsep}{0pt}\fcolorbox[rgb]{1.00,0.00,0.00}{1,1,1}{\strut ##1}}}
\expandafter\def\csname PY@tok@na\endcsname{\def\PY@tc##1{\textcolor[rgb]{0.49,0.56,0.16}{##1}}}

\def\PYZbs{\char`\\}
\def\PYZus{\char`\_}
\def\PYZob{\char`\{}
\def\PYZcb{\char`\}}
\def\PYZca{\char`\^}
\def\PYZam{\char`\&}
\def\PYZlt{\char`\<}
\def\PYZgt{\char`\>}
\def\PYZsh{\char`\#}
\def\PYZpc{\char`\%}
\def\PYZdl{\char`\$}
\def\PYZhy{\char`\-}
\def\PYZsq{\char`\'}
\def\PYZdq{\char`\"}
\def\PYZti{\char`\~}
% for compatibility with earlier versions
\def\PYZat{@}
\def\PYZlb{[}
\def\PYZrb{]}
\makeatother


    % Exact colors from NB
    \definecolor{incolor}{rgb}{0.0, 0.0, 0.5}
    \definecolor{outcolor}{rgb}{0.545, 0.0, 0.0}



    
    % Prevent overflowing lines due to hard-to-break entities
    \sloppy 
    % Setup hyperref package
    \hypersetup{
      breaklinks=true,  % so long urls are correctly broken across lines
      colorlinks=true,
      urlcolor=blue,
      linkcolor=darkorange,
      citecolor=darkgreen,
      }
    % Slightly bigger margins than the latex defaults
    
    \geometry{verbose,tmargin=1in,bmargin=1in,lmargin=1in,rmargin=1in}
    
    

    \begin{document}
    
    
    \maketitle
    
    

    
\subsection*{Derivation}\label{derivation}

If we expand a function \(f(x)\) about a point \(x_0\) with an offset of
\(\pm \delta/2\) using the Taylor series, we can obtain an expression
for the derivative

\begin{equation}
\left . \dfrac{ \mathrm{d} f}{\mathrm{d} x} \right |_{x=x_0} \approx \dfrac{ f(x+\delta/2) - f(x-\delta/2) }{\delta}
\end{equation}

Note that we ssumed that the offset \(\delta\) is very small, hence we
can drop the higher-order terms that appear in the Taylor expansion.

We use the Yee algorithm for Finite Difference Time Domain (FDTD), first
proposed by Kane Yee in 1966, which uses second-order central
differences. This algorithm can be summarized as follows: we first
obtain expressions for the electromagnetic field using Ampere's and
Faraday's Laws, then discretize them in space and time. From this, we
then solve for the fields - first the magnetic, then the electric - one
time step at a time starting from a set of initial conditions, until we
reach the final time step.

We begin using one-dimensional forms of the Ampere and Faraday laws in
electrodynamics

\begin{equation}
\mu \dfrac{\partial H_y}{\partial t} = \dfrac{\partial E_z}{\partial x}
\qquad\qquad
\epsilon \dfrac{\partial E_z}{\partial t} = \dfrac{\partial H_y}{\partial t}
\end{equation}

    Now we transform the continuous space and time coordinates into discrete
steps - the spatial coordinates will go as
\({ \Delta x, \Delta x + \Delta x,..., m\Delta x, ...}\), whereas the
temporal coordinates are written as
\({ \Delta t, \Delta t + \Delta t,..., q\Delta t, ...}\). From this
point, we will simply use \(m\) a the coordinate along x and \(q\) as
the time, with \(\Delta x\) and \(\Delta t\) referring to the time step
in the simulation.

For Faraday's Law at point \((q, m+1/2)\), we have

\begin{equation}
\mu \dfrac{1}{\Delta t} \left (H_y \Big |^{q+1/2}_{m+1/2} - H_y \Big |^{q-1/2}_{m+1/2} \right ) = \dfrac{1}{\Delta x} \left ( E_z \Big |^{q}_{m+1} - E_z \Big |^{q}_{m} \right )
\end{equation}

and solving for \(H_y \Big |^{q+1/2}_{m+1/2}\),

\begin{equation}
H_y \Big |^{q+1/2}_{m+1/2} = H_y \Big |^{q-1/2}_{m+1/2} + \dfrac{1}{\mu} \dfrac{\Delta t}{\Delta x} \left ( E_z \Big |^{q}_{m+1} - E_z \Big |^{q}_{m} \right )
\end{equation}

This is known as the updatee equation for the magnetic field. Using the
known value of \(H_y\), that is, at point \(q-1/2\), we use this, and
the electric field to update the magnetic field at the next time step.

Similarly for Ampere's Law, at point \((q+1/2, m)\), we have

\begin{equation}
\epsilon \dfrac{1}{\Delta t} \left ( E_z \Big |^{q+1}_{m} - E_z \Big |^{q}_{m} \right ) = \dfrac{1}{\Delta x} \left ( H_y \Big |^{q+1/2}_{m+1/2} - H_y \Big |^{q+1/2}_{m-1/2} \right )
\end{equation}

solving for \(E_z \Big |^{q+1}_{m}\),

\begin{equation}
E_z \Big |^{q+1}_{m} = E_z \Big |^{q}_{m} + \dfrac{1}{\epsilon} \dfrac{\Delta t}{\Delta x} \left ( H_y \Big |^{q+1/2}_{m+1/2} - H_y \Big |^{q+1/2}_{m-1/2} \right )
\end{equation}

The ratios \(\dfrac{1}{\epsilon} \dfrac{\Delta t}{\Delta x}\) and
\(\dfrac{1}{\mu} \dfrac{\Delta t}{\Delta x}\) can be written as

\begin{equation}
\dfrac{1}{\epsilon} \dfrac{\Delta t}{\Delta x} = \dfrac{\eta_0}{\epsilon_r} S_c
\end{equation}

\begin{equation}
\dfrac{1}{\mu} \dfrac{\Delta t}{\Delta x} = \dfrac{1}{\mu_r \eta_0} S_c
\end{equation}

where \(\epsilon_r\) and \(\mu_r\) are the relative permittivity and
permeability of the material in which the field propagates in,
\(\eta_0\) is the impedance of free space defined by

\begin{equation}
\eta_0 = \sqrt{\dfrac{\mu_0}{\epsilon_0}}
\end{equation}

and \(S_c\), the Courant number, is defined as the ratio

\begin{equation}
S_c = \dfrac{c \Delta t}{\Delta x}
\end{equation}

For this case, we will use a Courant number of unity.

    Given these definitions, we have the update equations for the magnetic
and electric fields as

\begin{equation}
H_y \Big |^{q+1/2}_{m+1/2} = H_y \Big |^{q-1/2}_{m+1/2} + \dfrac{1}{\mu_r \eta_0} S_c \left ( E_z \Big |^{q}_{m+1} - E_z \Big |^{q}_{m} \right )
\end{equation}

\begin{equation}
E_z \Big |^{q+1}_{m} = E_z \Big |^{q}_{m} + \dfrac{\eta_0}{\epsilon_r} S_c \left ( H_y \Big |^{q+1/2}_{m+1/2} - H_y \Big |^{q+1/2}_{m-1/2} \right )
\end{equation}

If the fields were in vacuum, \(\epsilon_r\) and \(\mu_r\) are equal to
one.

    Now we add a source to the simulation. We return to Ampere's Law, with
the term for the current density

\begin{equation}
\nabla \times \mathbf H = \mathbf J + \epsilon \dfrac{ \partial \mathbf E}{\partial t}
\end{equation}

For the one-dimensional case, we can rearrange this to

\begin{equation}
\dfrac{\partial E_z}{\partial t} = \dfrac{1}{\epsilon} \dfrac{\partial H_y}{\partial x} - \dfrac{1}{\epsilon} J
\end{equation}

And applying the same process as previously shown, the update equation
for \(E_z\) is

\begin{equation}
E_z \Big |^{q+1}_{m} = E_z \Big |^{q}_{m} + \dfrac{\eta_0}{\epsilon_r} S_c \left ( H_y \Big |^{q+1/2}_{m+1/2} - H_y \Big |^{q+1/2}_{m-1/2} \right ) - \dfrac{\Delta t}{\epsilon} J \Big |^{q+1/2}_{m}
\end{equation}

In order to be able to use the previous expression for the update
equation, we can write

\begin{equation}
E_z \Big |^{q+1}_{m} = E_z \Big |^{q+1}_{m} - \dfrac{\Delta t}{\epsilon} J \Big |^{q+1/2}_{m}
\end{equation}

after the code runs the previous line.

    \subsection*{Python Implementation}\label{python-implementation}

We begin with importing relevant libraries for displaying the fields
(matplotlib/pyplot), for the arrays (numpy), and for the function of the
source (math). A variable ``fignum'' is also initialized in order to
create multiple separate figures of the fields as it progresses in time.

    \begin{Verbatim}[commandchars=\\\{\}]
{\color{incolor}In [{\color{incolor}1}]:} \PY{o}{\PYZpc{}}\PY{k}{matplotlib} inline
        \PY{k+kn}{from} \PY{n+nn}{matplotlib} \PY{k}{import} \PY{n}{pyplot} \PY{k}{as} \PY{n}{plt}
        
        \PY{k+kn}{import} \PY{n+nn}{numpy} \PY{k}{as} \PY{n+nn}{np}
        \PY{k+kn}{import} \PY{n+nn}{math} \PY{k}{as} \PY{n+nn}{m}
        
        \PY{n}{fignum} \PY{o}{=} \PY{l+m+mi}{0}
\end{Verbatim}

    The size nx of the spatial domain is chosen arbitrarily. The source is
located at the center, and the width is also selected at random. Since
we are defining the domain directly using integers, \(\Delta x\) and
\(\Delta t\) are equal to 1; with the Courant number set to unity, this
also means that the speed of light is defined as 1.

A time delay is introduced in the source. In order to see the field as
it vanishes, the time domain size is set as the sum of the space domain
and the delay of the source.

    \begin{Verbatim}[commandchars=\\\{\}]
{\color{incolor}In [{\color{incolor}2}]:} \PY{n}{nx} \PY{o}{=} \PY{l+m+mi}{300}            \PY{c+c1}{\PYZsh{}spatial domain size}
        \PY{n}{srcori} \PY{o}{=} \PY{n+nb}{int}\PY{p}{(}\PY{n}{nx}\PY{o}{/}\PY{l+m+mi}{2}\PY{p}{)}  \PY{c+c1}{\PYZsh{}source location}
        \PY{n}{srcwid} \PY{o}{=} \PY{l+m+mi}{5}          \PY{c+c1}{\PYZsh{}source width}
        \PY{n}{srcdel} \PY{o}{=} \PY{l+m+mi}{15}\PY{o}{*}\PY{n}{srcwid}  \PY{c+c1}{\PYZsh{}source delay}
        \PY{n}{nt} \PY{o}{=} \PY{n}{nx}\PY{o}{+}\PY{n}{srcdel}      \PY{c+c1}{\PYZsh{}temporal domain size}
        
        \PY{n}{imp0} \PY{o}{=} \PY{l+m+mf}{337.0} \PY{c+c1}{\PYZsh{}impedance}
\end{Verbatim}

    Here, the electric and magnetic fields are initialized to zero, and an
array for the spatial coordinates from 0 to nx-1 is created.

    \begin{Verbatim}[commandchars=\\\{\}]
{\color{incolor}In [{\color{incolor}3}]:} \PY{n}{ez} \PY{o}{=} \PY{n}{np}\PY{o}{.}\PY{n}{zeros}\PY{p}{(}\PY{n}{nx}\PY{p}{)}
        \PY{n}{hy} \PY{o}{=} \PY{n}{np}\PY{o}{.}\PY{n}{zeros}\PY{p}{(}\PY{n}{nx}\PY{p}{)}
        \PY{n}{x} \PY{o}{=} \PY{n}{np}\PY{o}{.}\PY{n}{arange}\PY{p}{(}\PY{l+m+mi}{0}\PY{p}{,}\PY{n}{nx}\PY{o}{\PYZhy{}}\PY{l+m+mi}{1}\PY{p}{,}\PY{l+m+mi}{1}\PY{p}{)}
\end{Verbatim}

    The fields are calculated one time step at a time in this loop. The
source is defined as a Gaussian function.

The if statement from line 7 onwards is used to plot the fields. We note
that the first plot generated is not completely zero, but instead a very
small spike at the location of the additive source. As the source delay
is increase, the spike also decreases in magnitude.

    \begin{Verbatim}[commandchars=\\\{\}]
{\color{incolor}In [{\color{incolor}4}]:} \PY{n}{plt}\PY{o}{.}\PY{n}{hold}\PY{p}{(}\PY{k+kc}{True}\PY{p}{)}
        \PY{k}{for} \PY{n}{dt} \PY{o+ow}{in} \PY{n+nb}{range}\PY{p}{(}\PY{l+m+mi}{0}\PY{p}{,}\PY{n}{nt}\PY{p}{)}\PY{p}{:}
            \PY{n}{hy}\PY{p}{[}\PY{n}{x}\PY{p}{]} \PY{o}{=} \PY{n}{hy}\PY{p}{[}\PY{n}{x}\PY{p}{]} \PY{o}{+} \PY{p}{(}\PY{n}{ez}\PY{p}{[}\PY{n}{x}\PY{o}{+}\PY{l+m+mi}{1}\PY{p}{]}\PY{o}{\PYZhy{}}\PY{n}{ez}\PY{p}{[}\PY{n}{x}\PY{p}{]}\PY{p}{)}\PY{o}{/}\PY{n}{imp0}
            \PY{n}{ez}\PY{p}{[}\PY{n}{x}\PY{p}{]} \PY{o}{=} \PY{n}{ez}\PY{p}{[}\PY{n}{x}\PY{p}{]} \PY{o}{+} \PY{p}{(}\PY{n}{hy}\PY{p}{[}\PY{n}{x}\PY{p}{]}\PY{o}{\PYZhy{}}\PY{n}{hy}\PY{p}{[}\PY{n}{x}\PY{o}{\PYZhy{}}\PY{l+m+mi}{1}\PY{p}{]}\PY{p}{)}\PY{o}{*}\PY{n}{imp0}
        
            \PY{n}{ez}\PY{p}{[}\PY{n}{srcori}\PY{p}{]} \PY{o}{+}\PY{o}{=} \PY{n}{m}\PY{o}{.}\PY{n}{exp}\PY{p}{(}\PY{o}{\PYZhy{}}\PY{p}{(}\PY{p}{(}\PY{n}{dt}\PY{o}{\PYZhy{}}\PY{n}{srcdel}\PY{p}{)}\PY{o}{*}\PY{p}{(}\PY{n}{dt}\PY{o}{\PYZhy{}}\PY{n}{srcdel}\PY{p}{)}\PY{p}{)}\PY{o}{/}\PY{p}{(}\PY{n}{srcwid}\PY{o}{*}\PY{n}{srcwid}\PY{p}{)}\PY{p}{)}
            
            \PY{k}{if} \PY{p}{(}\PY{n}{dt} \PY{o}{\PYZpc{}} \PY{l+m+mi}{60} \PY{o}{==} \PY{l+m+mi}{0} \PY{o+ow}{or} \PY{n}{dt} \PY{o}{==} \PY{n}{nt}\PY{o}{\PYZhy{}}\PY{l+m+mi}{1}\PY{p}{)}\PY{p}{:}
                \PY{n}{fignum} \PY{o}{+}\PY{o}{=} \PY{l+m+mi}{1}
                \PY{n}{plt}\PY{o}{.}\PY{n}{figure}\PY{p}{(}\PY{n}{fignum}\PY{p}{)}
                \PY{n}{plt}\PY{o}{.}\PY{n}{xlabel}\PY{p}{(}\PY{l+s+s2}{\PYZdq{}}\PY{l+s+s2}{Position}\PY{l+s+s2}{\PYZdq{}}\PY{p}{)}
                \PY{n}{plt}\PY{o}{.}\PY{n}{ylabel}\PY{p}{(}\PY{l+s+s2}{\PYZdq{}}\PY{l+s+s2}{Field Amplitude}\PY{l+s+s2}{\PYZdq{}}\PY{p}{)}
                \PY{n}{plt}\PY{o}{.}\PY{n}{title}\PY{p}{(}\PY{l+s+s2}{\PYZdq{}}\PY{l+s+s2}{Field at t = }\PY{l+s+s2}{\PYZdq{}}\PY{o}{+} \PY{n+nb}{str}\PY{p}{(}\PY{n}{dt}\PY{p}{)}\PY{p}{)}        
                \PY{n}{plt}\PY{o}{.}\PY{n}{plot}\PY{p}{(}\PY{n}{ez}\PY{p}{,} \PY{n}{label}\PY{o}{=}\PY{l+s+s2}{\PYZdq{}}\PY{l+s+s2}{E\PYZhy{}field}\PY{l+s+s2}{\PYZdq{}}\PY{p}{)}
                
                \PY{k}{if} \PY{p}{(}\PY{n}{dt} \PY{o}{!=} \PY{n}{nt}\PY{o}{\PYZhy{}}\PY{l+m+mi}{1}\PY{p}{)}\PY{p}{:}
                    \PY{n}{plt}\PY{o}{.}\PY{n}{plot}\PY{p}{(}\PY{n}{hy}\PY{o}{*}\PY{n}{imp0}\PY{p}{,} \PY{n}{label}\PY{o}{=}\PY{l+s+s2}{\PYZdq{}}\PY{l+s+s2}{H\PYZhy{}field}\PY{l+s+s2}{\PYZdq{}}\PY{p}{)}
                    \PY{n}{plt}\PY{o}{.}\PY{n}{legend}\PY{p}{(}\PY{p}{)}
\end{Verbatim}

    \begin{center}
    \adjustimage{max size={0.5\linewidth}{0.5\paperheight}}{task0_files/task0_11_0.png}
    \end{center}
    { \hspace*{\fill} \\}
    
    \begin{center}
    \adjustimage{max size={0.5\linewidth}{0.5\paperheight}}{task0_files/task0_11_1.png}
    \end{center}
    { \hspace*{\fill} \\}
    
    \begin{center}
    \adjustimage{max size={0.5\linewidth}{0.5\paperheight}}{task0_files/task0_11_2.png}
    \end{center}
    { \hspace*{\fill} \\}
    
    \begin{center}
    \adjustimage{max size={0.5\linewidth}{0.5\paperheight}}{task0_files/task0_11_3.png}
    \end{center}
    { \hspace*{\fill} \\}
    
    \begin{center}
    \adjustimage{max size={0.5\linewidth}{0.5\paperheight}}{task0_files/task0_11_4.png}
    \end{center}
    { \hspace*{\fill} \\}
    
    \begin{center}
    \adjustimage{max size={0.5\linewidth}{0.5\paperheight}}{task0_files/task0_11_5.png}
    \end{center}
    { \hspace*{\fill} \\}
    
    \begin{center}
    \adjustimage{max size={0.5\linewidth}{0.5\paperheight}}{task0_files/task0_11_6.png}
    \end{center}
    { \hspace*{\fill} \\}
    
    \begin{center}
    \adjustimage{max size={0.5\linewidth}{0.5\paperheight}}{task0_files/task0_11_7.png}
    \end{center}
    { \hspace*{\fill} \\}
    
    \subsection*{References}\label{references}

Schneider, J. B. \emph{Understanding the finite-difference time-domain
method}. School of electrical engineering and computer science
Washington State University.--URL: http://www. Eecs. Wsu. Edu/∼
schneidj/ufdtd/(request data: 29.11. 2012) (2010).


    % Add a bibliography block to the postdoc
    
    
    
    \end{document}
